\documentclass[letterpaper,oneside]{book}

\usepackage{amsmath}
\usepackage{amsthm}
\usepackage{amssymb}
\usepackage{textcomp}
\usepackage{gensymb}

\providecommand{\ol}{\overline}
\providecommand{\ul}{\underline}
\providecommand{\wt}{\widetilde}
\providecommand{\wh}{\widehat}
\providecommand{\eps}{\varepsilon}
\providecommand{\half}{\frac{1}{2}}
\providecommand{\inv}{^{-1}}
\newcommand{\dang}{\measuredangle} %% Directed angle
\providecommand{\CC}{\mathbb C}
\providecommand{\FF}{\mathbb F}
\providecommand{\NN}{\mathbb N}
\providecommand{\QQ}{\mathbb Q}
\providecommand{\RR}{\mathbb R}
\providecommand{\ZZ}{\mathbb Z}
\providecommand{\ts}{\textsuperscript}
\providecommand{\dg}{^\circ}
\providecommand{\ii}{\item}
\providecommand{\defeq}{\coloneqq}
\DeclareMathOperator*{\lcm}{lcm}
\DeclareMathOperator*{\argmin}{arg min}
\DeclareMathOperator*{\argmax}{arg max}
\providecommand{\hrulebar}{\par
\hspace{\fill}\rule{0.95\linewidth}{.7pt}\hspace{\fill}
\par\nointerlineskip \vspace{\baselineskip}}

\setlength{\parindent}{0pt}
\begin{document}
\title{CNCM Problem of the Day Solutions}
\author{Ryder Pham}
\maketitle

\section*{6 August 2021}

Notice that the expected number of dollars Tommy expects to win is equivalent to the following infinite series:

$$\frac16 \sum_{n=0}^\infty n^2\left(\frac56\right)^n.$$

Define
$$f(x) = \sum_{n=0}^\infty n^2x^n$$
where we want to find the value of $f(5/6)$.
Then
\begin{align*}
    f(x)   & = x\sum_{n=0}^\infty n^2x^{n-1}                                             \\
           & = x\frac{d}{dx}\left[\sum_{n=0}^\infty nx^n\right]                          \\
           & = x\frac{d}{dx}\left[x\sum_{n=0}^\infty nx^{n-1}\right]                     \\
           & = x\frac{d}{dx}\left[x\frac{d}{dx}\left[\sum_{n=0}^\infty x^n\right]\right] \\
           & = x\frac{d}{dx}\left[x\frac{d}{dx}\left[\frac{1}{1-x}\right]\right]         \\
           & = x\frac{d}{dx}\left[\frac{x}{(1-x)^2}\right]                               \\
           & = x\left[\frac{(1-x)^2+2x(1-x)}{(1-x)^4}\right]                             \\
           & = x\left[\frac{(1-x)+2x}{(1-x)^3}\right]                                    \\
           & = x\left[\frac{1+x}{(1-x)^3}\right]                                         \\
    f(5/6) & = \frac56 \left[\frac{1+5/6}{(1-5/6)^3}\right]                              \\
           & = \frac56 \cdot \frac{11}{6} \cdot \frac{6^3}{1}                            \\
           & = 330.
\end{align*}
Then our final answer is $\frac16 f(\frac56) = \frac{330}{6}= \fbox{55}$.

\section*{11 August 2021}
Here let $R_n$ denote the remaining water after the $n$-th pour.
\begin{align*}
    R_0 & = 1                                           \\
    R_1 & = \left(1-\frac{1}{2}\right)R_0 = \frac{1}{2} \\
    R_2 & = \left(1-\frac{1}{3}\right)R_1 = \frac{1}{3} \\
    R_3 & = \left(1-\frac{1}{4}\right)R_2 = \frac{1}{4}
\end{align*}
Therefore we can assume by Engineer's Induction that $R_n = \frac{1}{n+1}.$ Hence $R_9 = \frac{1}{10}$ for a final answer of $\fbox{9}.$

\section*{12 August 2021}
For a two-game block, there is a probability of $\displaystyle \frac34\cdot\frac14 + \frac14\cdot\frac34 = \frac38$ of the entire match ending then and there. The only other outcome after two games is a tie, since each game must declare a winner, and this happens with probability $5/8.$ Thus the expected number of games in a match is the following:
\begin{align*}
    \frac38\cdot2 + \frac38\cdot\frac58\cdot4+ \frac38\cdot\left(\frac58\right)^2\cdot6 + \cdots
     & = \sum_{n=0}^{\infty} 2(n+1)\cdot\frac38\cdot\left(\frac58\right)^n                                        \\
     & = \frac34 \sum_{n=0}^{\infty} (n+1)\left(\frac58\right)^n                                                  \\
     & = \frac34 \sum_{n=0}^{\infty} n\left(\frac58\right)^n + \frac34\sum_{n=0}^{\infty} \left(\frac58\right)^n.
\end{align*}

Define $f(x) = \sum_{n=0}^\infty nx^n.$ We would like to find the value of $f(5/8).$ Note that
\begin{align*}
    f(x) & = x\sum_{n=0}^{\infty} nx^{n-1}            \\
         & = x \frac{d}{dx}\sum_{n=0}^{\infty} x^n    \\
         & = x \frac{d}{dx}\left[\frac{1}{1-x}\right] \\
         & = \frac{x}{(1-x)^2}.
\end{align*}
Thus $\displaystyle f(5/8)= \frac{5/8}{(1-5/8)^2} = \frac58 \cdot \frac{8^2}{3^2} = \frac{40}{9}.$ Then our original sum becomes
\begin{align*}
    \frac34 \sum_{n=0}^{\infty} n\left(\frac58\right)^n + \frac34\sum_{n=0}^{\infty} \left(\frac58\right)^n & = \frac34\cdot\frac{40}{9}+\frac34\cdot\frac{1}{1-5/8} \\
                                                                                                            & = \frac{10}{3} + \frac{3}{4}\cdot\frac{8}{3}           \\
                                                                                                            & = \frac{10}{3} + 2                                     \\
                                                                                                            & = \frac{16}{3}.
\end{align*}
Our final answer is $160+3 = \fbox{163}.$

\section*{26 August 2021}
Our recurrence relation is
$$7a_{n}=-a_{n-1}+8a_{n-2}.$$
By simple calulations we determine that $a_1=25.$
Note that the recurrence is linear and homogenous. Its characteristic equation is
\begin{align*}
    7r^2+r-8    & =0                \\
    (7r+8)(r-1) & =0                \\
    r_{1,2}     & = -\frac{8}{7},1.
\end{align*}
So by some theorem (idk) $a_n=\alpha_1\left(-\frac{8}{7}\right)^n+\alpha_2(1)^n=\alpha_1\left(-\frac{8}{7}\right)^n+\alpha_2$ is a solution. To find $\alpha_1,\alpha_2$ we must solve the following system:
$$
    \begin{cases}
        a_0 = \alpha_1+\alpha_2=4 \\
        a_1 = -\frac{8}{7}\alpha_1+\alpha_2=25.
    \end{cases}
$$ Solving this gets us $(\alpha_1,\alpha_2) = (-49/5, 69/5).$ Thus what we have left to evaluate is
\begin{align*}
    a_7 & = -\frac{49}{5}\left(-\frac{8}{7}\right)^7+\frac{69}{5} \\
    & = \frac{49}{5}\left(\frac{8}{7}\right)^7+\frac{69}{5} \\
    & = \frac{1}{5}\cdot\frac{8^7}{7^5}+\frac{69}{5} \\
    & = \frac{8^7+69\cdot7^5}{5\cdot7^5}\\
    & = \frac{8^7+69\cdot16807}{5\cdot7^5}\\
    & = \frac{2097152+69\cdot16807}{5\cdot7^5}\\
    & = \frac{2097152+1159683}{5\cdot7^5}\\
    & = \frac{3256835}{5\cdot7^5}\\
    & = \frac{651367}{16807}.\\
\end{align*}
Therefore our final answer is $651367+16807+28795=\fbox{696969}.$
\end{document}