\documentclass[letterpaper,oneside]{scrartcl}

\usepackage[sexy]{evan}
\usepackage{amsmath}
\usepackage{amsthm}
\usepackage{amssymb}
\usepackage{textcomp}
\usepackage{gensymb}

\providecommand{\ol}{\overline}
\providecommand{\ul}{\underline}
\providecommand{\wt}{\widetilde}
\providecommand{\wh}{\widehat}
\providecommand{\eps}{\varepsilon}
\providecommand{\half}{\frac{1}{2}}
\providecommand{\inv}{^{-1}}
\providecommand{\CC}{\mathbb C}
\providecommand{\FF}{\mathbb F}
\providecommand{\NN}{\mathbb N}
\providecommand{\QQ}{\mathbb Q}
\providecommand{\RR}{\mathbb R}
\providecommand{\ZZ}{\mathbb Z}
\providecommand{\ts}{\textsuperscript}
\providecommand{\dg}{^\circ}
\providecommand{\ii}{\item}
\providecommand{\defeq}{\coloneqq}
\providecommand{\hrulebar}{\par
\hspace{\fill}\rule{0.95\linewidth}{.7pt}\hspace{\fill}
\par\nointerlineskip \vspace{\baselineskip}}

\setlength{\parindent}{0pt}
\begin{document}
\title{CNCM Problem of the Day Solutions}
\author{Ryder Pham}
\maketitle
\newpage
\section*{6 August 2021}

Notice that the expected number of dollars Tommy expects to win is equivalent to the following infinite series:

$$\frac16 \sum_{n=0}^\infty n^2\left(\frac56\right)^n.$$

Define
$$f(x) = \sum_{n=0}^\infty n^2x^n$$
where we want to find the value of $f(5/6)$.
Then
\begin{align*}
    f(x)   & = x\sum_{n=0}^\infty n^2x^{n-1}                                             \\
           & = x\frac{d}{dx}\left[\sum_{n=0}^\infty nx^n\right]                          \\
           & = x\frac{d}{dx}\left[x\sum_{n=0}^\infty nx^{n-1}\right]                     \\
           & = x\frac{d}{dx}\left[x\frac{d}{dx}\left[\sum_{n=0}^\infty x^n\right]\right] \\
           & = x\frac{d}{dx}\left[x\frac{d}{dx}\left[\frac{1}{1-x}\right]\right]         \\
           & = x\frac{d}{dx}\left[\frac{x}{(1-x)^2}\right]                               \\
           & = x\left[\frac{(1-x)^2+2x(1-x)}{(1-x)^4}\right]                             \\
           & = x\left[\frac{(1-x)+2x}{(1-x)^3}\right]                                    \\
           & = x\left[\frac{1+x}{(1-x)^3}\right]                                         \\
    f(5/6) & = \frac56 \left[\frac{1+5/6}{(1-5/6)^3}\right]                              \\
           & = \frac56 \cdot \frac{11}{6} \cdot \frac{6^3}{1}                            \\
           & = 330.
\end{align*}
Then our final answer is $\frac16 f(\frac56) = \frac{330}{6}= \fbox{55}$.

\section*{11 August 2021}
Here let $R_n$ denote the remaining water after the $n$-th pour.
\begin{align*}
    R_0 & = 1                                           \\
    R_1 & = \left(1-\frac{1}{2}\right)R_0 = \frac{1}{2} \\
    R_2 & = \left(1-\frac{1}{3}\right)R_1 = \frac{1}{3} \\
    R_3 & = \left(1-\frac{1}{4}\right)R_2 = \frac{1}{4}
\end{align*}
Therefore we can assume by Engineer's Induction that $R_n = \frac{1}{n+1}.$ Hence $R_9 = \frac{1}{10}$ for a final answer of $\fbox{9}.$

\section*{12 August 2021}
For a two-game block, there is a probability of $\displaystyle \frac34\cdot\frac14 + \frac14\cdot\frac34 = \frac38$ of the entire match ending then and there. The only other outcome after two games is a tie, since each game must declare a winner, and this happens with probability $5/8.$ Thus the expected number of games in a match is the following:
\begin{align*}
    \frac38\cdot2 + \frac38\cdot\frac58\cdot4+ \frac38\cdot\left(\frac58\right)^2\cdot6 + \cdots
     & = \sum_{n=0}^{\infty} 2(n+1)\cdot\frac38\cdot\left(\frac58\right)^n                                        \\
     & = \frac34 \sum_{n=0}^{\infty} (n+1)\left(\frac58\right)^n                                                  \\
     & = \frac34 \sum_{n=0}^{\infty} n\left(\frac58\right)^n + \frac34\sum_{n=0}^{\infty} \left(\frac58\right)^n.
\end{align*}

Define $f(x) = \sum_{n=0}^\infty nx^n.$ We would like to find the value of $f(5/8).$ Note that
\begin{align*}
    f(x) & = x\sum_{n=0}^{\infty} nx^{n-1}            \\
         & = x \frac{d}{dx}\sum_{n=0}^{\infty} x^n    \\
         & = x \frac{d}{dx}\left[\frac{1}{1-x}\right] \\
         & = \frac{x}{(1-x)^2}.
\end{align*}
Thus $\displaystyle f(5/8)= \frac{5/8}{(1-5/8)^2} = \frac58 \cdot \frac{8^2}{3^2} = \frac{40}{9}.$ Then our original sum becomes
\begin{align*}
    \frac34 \sum_{n=0}^{\infty} n\left(\frac58\right)^n + \frac34\sum_{n=0}^{\infty} \left(\frac58\right)^n & = \frac34\cdot\frac{40}{9}+\frac34\cdot\frac{1}{1-5/8} \\
                                                                                                            & = \frac{10}{3} + \frac{3}{4}\cdot\frac{8}{3}           \\
                                                                                                            & = \frac{10}{3} + 2                                     \\
                                                                                                            & = \frac{16}{3}.
\end{align*}
Our final answer is $160+3 = \fbox{163}.$

\section*{26 August 2021}
Our recurrence relation is
$$7a_{n}=-a_{n-1}+8a_{n-2}.$$
By simple calulations we determine that $a_1=25.$
Note that the recurrence is linear and homogenous. Its characteristic equation is
\begin{align*}
    7r^2+r-8    & =0                \\
    (7r+8)(r-1) & =0                \\
    r_{1,2}     & = -\frac{8}{7},1.
\end{align*}
So by some theorem (idk) $a_n=\alpha_1\left(-\frac{8}{7}\right)^n+\alpha_2(1)^n=\alpha_1\left(-\frac{8}{7}\right)^n+\alpha_2$ is a solution. To find $\alpha_1,\alpha_2$ we must solve the following system:
$$
    \begin{cases}
        a_0 = \alpha_1+\alpha_2=4 \\
        a_1 = -\frac{8}{7}\alpha_1+\alpha_2=25.
    \end{cases}
$$ Solving this gets us $(\alpha_1,\alpha_2) = (-49/5, 69/5).$ Thus what we have left to evaluate is
\begin{align*}
    a_7 & = -\frac{49}{5}\left(-\frac{8}{7}\right)^7+\frac{69}{5} \\
        & = \frac{49}{5}\left(\frac{8}{7}\right)^7+\frac{69}{5}   \\
        & = \frac{1}{5}\cdot\frac{8^7}{7^5}+\frac{69}{5}          \\
        & = \frac{8^7+69\cdot7^5}{5\cdot7^5}                      \\
        & = \frac{8^7+69\cdot16807}{5\cdot7^5}                    \\
        & = \frac{2097152+69\cdot16807}{5\cdot7^5}                \\
        & = \frac{2097152+1159683}{5\cdot7^5}                     \\
        & = \frac{3256835}{5\cdot7^5}                             \\
        & = \frac{651367}{16807}.                                 \\
\end{align*}
Therefore our final answer is $651367+16807+28795=\fbox{696969}.$
\newpage
\section*{2 September 2021}
Let $BD =x.$ By the Angle Bisector Theorem
$$\frac{8}{12} = \frac{x}{10-x}.$$
Solving for $x$ gives us $x=4.$ Thus $BD=4$ and $CD=6.$ By Stewart's Theorem on $\triangle ABC$ we have
\begin{align*}
    b^2m+c^2n            & =a(d^2+mn)       \\
    8^2\cdot6+12^2\cdot4 & =10(d^2+6\cdot4) \\
    384+576              & =10d^2+240       \\
    d^2                  & =72              \\
    AD                   & =6\sqrt2.
\end{align*}
We will now find $BD'$. Applying Stewart's Theorem again on $\triangle ABD'$ gives us
\begin{align*}
    8^2\cdot2\sqrt2+c^2\cdot6\sqrt2 & =8\sqrt2(4^2+24) \\
    128+6c^2                        & =128+192         \\
    c^2                             & =32              \\
    BD'                             & =4\sqrt2.
\end{align*}
Similarly to find $CD'$ we apply Stewart's Theorem a third time on $\triangle ACD',$ which gives us
\begin{align*}
    b^2m+c^2n                        & =a(d^2+mn)       \\
    12^2\cdot2\sqrt2+c^2\cdot6\sqrt2 & =8\sqrt2(6^2+24) \\
    288+6c^2                         & =288+192         \\
    c^2                              & =32              \\
    CD'                              & =4\sqrt2.
\end{align*}
Thus $BD'\cdot CD' = (4\sqrt2)^2=\fbox{32}.$
\newpage
\section*{6 September 2021}
Let \(PY=x\)  and \(QY=y.\)
Note that the area of \(\triangle XYZ\) (henceforth denoted \([XYZ]\)) is \(\frac{1}{2}\cdot12\cdot2004.\)
Also note that \([XYZ]=[XPY]+[YPZ].\) It follows that
\begin{align*}
    [XYZ]                   & =[XPY]+[YPZ]                                                            \\
    \frac12\cdot12\cdot2004 & = \frac12\cdot12\cdot\frac{x}{2}+\frac12\cdot2004\cdot\frac{x\sqrt3}{2} \\
    12\cdot2004             & =6x+1002\sqrt3x                                                         \\
    x                       & = \frac{2\cdot2004}{1+167\sqrt3}                                        \\
\end{align*}
Similarly we have
\begin{align*}
    [XYZ]                   & =[XQY]+[YQZ]                                                            \\
    \frac12\cdot12\cdot2004 & = \frac12\cdot12\cdot\frac{y\sqrt3}{2}+\frac12\cdot2004\cdot\frac{y}{2} \\
    12\cdot2004             & =6\sqrt3y+1002y                                                         \\
    y                       & =\frac{2\cdot2004}{167+\sqrt3}                                          \\
\end{align*}
Thus
\begin{align*}
    (PY+YZ)(QY+XY) & =\left(\frac{2\cdot12\cdot167}{1+167\sqrt3}+12\cdot167\right)\left(\frac{2\cdot12\cdot167}{167+\sqrt3}+12\right)     \\
                   & =12\cdot167\cdot12\left(\frac{2+1+167\sqrt3}{1+167\sqrt3}\right)\left(\frac{2\cdot167+167+\sqrt3}{167+\sqrt3}\right) \\
                   & =12\cdot167\cdot12\left(\frac{3+167\sqrt3}{1+167\sqrt3}\right)\left(\frac{3\cdot167+\sqrt3}{167+\sqrt3}\right)       \\
                   & =12\cdot167\cdot12\left(\frac{167^2\cdot3\sqrt3+12\cdot167+3\sqrt3}{167^2\sqrt3+4\cdot167+\sqrt3}\right)             \\
                   & =12\cdot167\cdot12\cdot3\left(\frac{167^2\sqrt3+4\cdot167+\sqrt3}{167^2\sqrt3+4\cdot167+\sqrt3}\right)               \\
                   & =\fbox{72144}.
\end{align*}
\section*{8 September 2021}
We have \(f(x)=\log_2(x+\underbrace{\log_2(x+\log_2(x+\cdots))}_{f(x)}) = \log_2(x+f(x))\). Solving for \(x\) gives us \(x=2^{f(x)}-f(x).\) Thus
\[f^{-1}(x)=2^x-x.\]
It follows that
\begin{align*}
    \sum_{k=2}^{10} \left[2^k-k\right] & = \sum_{k=2}^{10}2^k-\sum_{k=2}^{10}k               \\
                                       & = \sum_{k=0}^{8} 2^{k+2}-\sum_{k=0}^{8}(k+2)        \\
                                       & = 4\cdot\frac{2^9-1}{2-1}-\frac{8\cdot9}{2}-2\cdot9 \\
                                       & = 4(511)-36-18                                      \\
                                       & = \fbox{1990}.
\end{align*}

\section*{17 September 2021}
Right off the bat, we can ignore dividing by 1 since every number is divisible by 1.
\begin{claim*}
    All \(n=11(2k+1), k \in \ZZ\) fail.
\end{claim*}
\begin{proof}
    Note that
    \begin{align*}
        n & \equiv 22k+11 \equiv 1 & \pmod{2}   \\
        n & \equiv 0               & \pmod{11}.
    \end{align*}
    Since \(1>0\), this is a decerasing sequence.
\end{proof}
\begin{claim*}
    The remainders of 11, 22, 121, and 242 always follow a non-decreasing sequence except for \(n \in [121,131]\).
\end{claim*}
\begin{proof}
    We will start with \((11,22)\). We can represent \(n = 22q_1+r_1\) with \(0 \leq r_1 < 22\). Then
    \begin{align*}
        n & = 22q_1 + r_1          \\
          & = 11(2q_1) + r_1       \\
          & = 11(2q_1) + 11k + r_2 \\
          & = 11(2q_1+k) + r_2,
    \end{align*}
    where \(0\leq r_2 < 11\). If \(r_1\geq 11,\) then \(k=1\) and \(r_1 > r_2\). If \(r_1 < 11\), then \(k=0\) and \(r_1=r_2\). Thus \(n\pmod{22}>n\pmod{11}\). Very similar arguments apply between \((11,121),(11,242),\) and \((121,242)\) since, in each pair, one is a multiple of the other. Additionally, since each pair cannot decrease the remainder of \(n\), together they must form a non-decreasing sequence.
    
    However, we are not done, since we must still consider \((22,121)\) because 121 is not a multiple of 22. Write \(n=121q_1+r_1\) with \(0 \leq r_1 < 121\). Note that if \(q_1\) is even, we are done since we can follow a very similar line of reasoning as above to show that \(r_1 \geq r_2\). We will now deal with the case if \(q_1\) is odd. Note that
    \begin{align*}
        n & = 121q_1+r_1         \\
          & = 121(2k+1)+r_1      \\
          & = 2\cdot121k+121+r_1 \\
          & = 22(11k+5)+11+r_1,
    \end{align*}
    where \(k\geq0\). If \(r_1\geq11\), then \(r_1 > r_2\), so we are good here. However, if \(r_1 <11\), then \(r_2=11+r_1>r_1\), so this case fails here. The only possible odd value that \(q_1\) can be and still have \(n\leq 242\) is 1, so \(n \in \{121,122,...,131\}\) all fail.
\end{proof}
Now, counting up how many \(n\) fail and subtracting from 242 we get \(242-\underbrace{11}_{\text{first claim}}-\underbrace{11}_{\text{second claim}}+\underbrace{1}_{\text{double counting } 121} = \fbox{221}\).

\section*{20 September 2021}
First note that \(a_k=\frac{k(k+1)}{2}\). We will now manipulate our product as follows:
\begin{align*}
    \prod_{t=2}^{2021} \frac{a_t}{a_t-1} &= \prod_{t=2}^{2021} \frac{\tfrac{t(t+1)}{2}}{\tfrac{t(t+1)}{2}-1}\\
    &= \prod_{t=2}^{2021} \frac{\tfrac{t(t+1)}{2}}{\tfrac{t(t+1)-2}{2}}\\
    &= \prod_{t=2}^{2021} \frac{t(t+1)}{t(t+1)-2}\\
    &= \prod_{t=2}^{2021} \frac{t(t+1)}{(t+2)(t-1)}\\
    &= \frac{2\cdot3}{4\cdot1}\cdot\frac{3\cdot4}{5\cdot2}\cdot\frac{4\cdot5}{6\cdot3}\cdots\frac{2021\cdot2022}{2023\cdot2020}\\
    &= \frac{3}{1}\cdot\frac{2021}{2023}\\
    &= \frac{6063}{2023}.
\end{align*}
Hence our final answer is \(6063+2023 = \fbox{8086}\).

\section*{6 October 2021 INCOMPLETE SOLUTION}
Note that for \(x,y=0,\) we have \(f(0) = -1\). For \(y=1\) and \(x\) free, we have
\[f(x+1)-f(x)=x+2.\]
Making the substitution \(x \mapsto x+1,\) we obtain 
\[f(x+2)-f(x+1)=x+3.\]
Then, for any integer \(k\), making the substitution \(x \mapsto x+k\) gets us
\[f(x+k)-f(x+k-1)=x+k+1.\]
Adding up all the equations from \(k=1\) to \(k=n\) gives us (this equation does not match what I have on paper)
\[f(x+n)-f(x)=nx+\frac{n(n+1)}{2}. \]

\section*{25 October 2021}
Denote by \(D\) the foot of the altitude from \(M\) to \(OC\), and let \(x = MD\). Note that since \(AM = 3, AC = 6\), and \(\angle AMC = 90\degree\), we have that \(MC = 3\sqrt3\) by Pythagoras on \(\triangle AMC\). Similarly, by Pythagoras on \(\triangle AMO\), we have that \(MO = 6\sqrt2\). Note that \(OD + DC = 9\). Thus, by using Pythagoras on \(\triangle MOD\) and \(\triangle MCD\), we have the following equation:
\begin{align*}
    \sqrt{(6\sqrt2)^2-x^2}+\sqrt{(3\sqrt3)^2-x^2} &= 9 \\
    \sqrt{72-x^2}+\sqrt{27-x^2} &= 9.
\end{align*}
We can make the substitution \(u^2=27-x^2\). Note that \(u^2+45=72-x^2\). Now we have
\begin{align*}
    \sqrt{u^2+45} + \sqrt{u^2} &= 9 \\
    u^2+45 &= (9-u)^2 \\
    45 &= 81-18u \\
    u &= 2.
\end{align*}
Hence \(x^2 = 27-u^2 = \fbox{23}.\)
\end{document}