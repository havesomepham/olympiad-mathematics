\documentclass[letterpaper,oneside]{scrartcl}

\usepackage[sexy]{evan}
\usepackage{amsmath}
\usepackage{amsthm}
\usepackage{amssymb}
\usepackage{textcomp}
\usepackage{gensymb}
\usepackage{xcomment}

\usepackage{asymptote}
\begin{document}
\title{Random Problems}
\author{Ryder Pham}
\maketitle

\textbf{Note:} The symbol $\dang$ refers to the directed angle in this text.

\begin{problem*}
  [OTIS Excerpts \#7]
  Determine, with proof, the smallest positive integer $c$ such that for any positive integer $n$, the decimal representation of the number $c^n + 2014$ has digits all less than $5$.
\end{problem*}
\begin{proof}
  We claim that $c=10.$ We know this value works because for $n\geq 1, c^n \in \{10,100,1000,\dots\},$ and since all digits from the 10s and to the left are less than 4, adding 1 to them will not violate our digit condition. We will now check that this is the smallest possible value for $c.$
  \begin{itemize}
    \ii $c=1$ fails at $n=1$ since $1+4=5.$
    \ii $c=2$ fails at $n=1$ since $2+4=6.$
    \ii $c=3$ fails at $n=1$ since $3+4=7.$
    \ii $c=4$ fails at $n=1$ since $4+4=8.$
    \ii $c=5$ fails at $n=1$ since $5+4=9.$
    \ii $c=6$ fails at $n=2$ since $36+2014=2050.$
    \ii $c=7$ fails at $n=2$ since $49+2014=2063.$
    \ii $c=8$ fails at $n=2$ since $64+2014=2078.$
    \ii $c=9$ fails at $n=2$ since $81+2014=2095.$
  \end{itemize}
  Since every value of $c$ less than $10$ fails, we are done.
\end{proof}
\begin{problem*}
  [OTIS Excerpts \#77, HMMT Februrary 2013]
  Values $a_1,\dots,a_{2013}$ are chosen independently and at random from the set $\{1,\dots,2013\}.$ What is the expected number of distinct values in the set $\{a_1,\dots,a_{2013}\}?$
\end{problem*}
\begin{soln}
  Let $P$ be the number of distinct values in ${a_1,\dots,a_{2013}}$, and for each $i=1,2,\dots,2013$ let
  $$
    P_i :=
    \begin{cases}
      1 & \text{if } a_i \neq a_j \text{ for all } j < i \\
      0 & \text{otherwise.}
    \end{cases}
  $$
  It is clear that $P = P_1+\cdots+P_{2013}.$ Thus it follows that
  \begin{align*}
    E[P] & =E[P_1]+E[P_2]+\cdots+E[P_{2013}]                           \\
         & = 1+(1-1/2013)+\cdots+(1-1/2013)^{2012}                     \\
         & = \frac{1-(2012/2013)^{2013}}{1-2012/2013}                  \\
         & = 2013\left(1-\left(\frac{2012}{2013}\right)^{2013}\right).
  \end{align*}
\end{soln}
\begin{problem*}
  [100 Geometry Problems \#8]
  Let $ABC$ be a triangle with $\angle CAB$ a right angle. The point $L$ lies on the side $BC$ between $B$ and $C$. The circle $BAL$ meets the line $AC$ again at $M$ and the circle $CAL$ meets the line $AB$ again at $N.$ Prove that $L, M,$ and $N$ lie on a straight line.
\end{problem*}
\begin{proof}
  Since $ANLC$ and $ALBM$ are cyclic quadrilaterals, $\angle CAN = \angle CLN = \angle 90\degree = \angle BAM = \angle BLM.$ Since $\angle CLN + \angle BLN = 180\degree,$ we have $\angle BLN = \angle BLM = 90\degree,$ as desired.
\end{proof}
\begin{problem*}
  [100 Geometry Problems \#11]
  A closed planar shape is said to be equiable if the numerical values of its perimeter and area are the same. For example, a square with side length 4 is equiable since its perimeter and area are both 16. Show that any closed shape in the plane can be dilated to become equiable. (A dilation is an affine transformation in which a shape is stretched or shrunk. In other words, if $\mathcal{A}$ is a dilated version of $\mathcal{B}$ then $\mathcal{A}$ is similar to $\mathcal{B}$.)
\end{problem*}
\begin{proof}
  Note that for any scaling of the perimeter by a factor of $k$, the area increases by a factor of $k^2.$ It is not hard to see that by making the perimeter arbitrarily large, at some point the area must be larger than the perimeter, and by making the perimeter arbitrarily small, the area must be smaller than the perimeter. Thus by the Intermediate Value Theorem there must be a scale factor $k$ such that the perimeter equals the area.
\end{proof}
\begin{problem*}
  [100 Geometry Problems \#13]
  Points $A$ and $B$ are located on circle $\Gamma$, and point $C$ is an arbitrary point in the interior of $\Gamma$. Extend $AC$ and $BC$ past $C$ so that they hit $\Gamma$ at $M$ and $N$ respectively. Let $X$ denote the foot of the perpendicular from $M$ to $BN$, and let $Y$ denote the foot of the perpendicular from $N$ to $AM$. Prove that $AB \parallel XY$ .
\end{problem*}
\begin{proof}
  It suffices to show that $\triangle ABC \sim \triangle YXC,$ as this would prove that $AB \parallel XY.$ Note that $NXYM$ is a cyclic quadrilateral because $\dang NXM = 90\degree = \dang NYM.$ By angle chasing we get
  $$\dang ABC = \dang ABN = \dang AMN  = \dang YMN = \dang YXN = \dang YXC.$$
  We know $\dang ACB = \dang YCX$ by vertical angles, hence $\triangle ABC \sim \triangle YXC$ by $AA\sim.$ This completes the proof.
\end{proof}
\begin{problem*}
  [100 Geometry Problem \#14, AIME 2007]
  Square $ABCD$ has side length 13, and points $E$ and $F$ are exterior to the square such that $BE = DF = 5$ and $AE = CF = 12$. Find $EF^2$.
\end{problem*}
\begin{soln}
  Extend $BE$ and $CF$ to meet at $G$ and extend $AE$ and $DF$ to meet at $H.$ Note by symmetry, $FGHE$ is a square of sidelength $12+5=17.$ Thus $EF^2 = 17^2+17^2 = \fbox{578}.$
\end{soln}
\begin{problem*}
  [100 Geometry Problems \#15]
  Let $\Gamma$ be the circumcircle of $\triangle ABC,$ and let $D,E,F$ be the midpoints of arcs $AB,BC,CA,$ respectively. Prove that $DF\perp AE.$
\end{problem*}
\begin{proof}
  Denote by $I$ the incenter of $\triangle ABC.$ By the Incenter-Excenter Lemma, $I$ lies on $AE.$ Also by the Lemma, $D$ and $F$ are the circumcenters of $(AIB)$ and $(AIC),$ respectively. The radical axis of these two circles is $AI,$ thus $AI \perp DF \Longrightarrow AE \perp DF.$
\end{proof}
\begin{problem*}
  [Andrews, NT, Problem 1-1.7]
  Denote by $F_n$ the $n$-th Fibonacci Number. Prove that
  \[F_1+F_2+F_3+\dots+F_n=F_{n+2}-1.\]
\end{problem*}
\begin{proof}
  We will use induction.\\
  \indent\textit{Base Case:} \(n=1.\) Notice that \(1=F_1=F_3-1=2-1.\)\\
  \indent\textit{Induction Hypothesis:} Assume our desired equation is true for all \(n \leq k\). We will now show that our desired equation holds for \(n = k+1.\) Note that
  \[(F_1+\dots+F_k)+F_{k+1}=F_{k+2}-1+F_{k+1}=F_{k+3}-1,\]
  where the first equality holds by our Induction Hypothesis and the second holds by the definition of the Fibonacci Sequence. This concludes the proof.
\end{proof}
\newpage
\begin{problem*}
  [Andrews, NT, Problem 1-1.8]
  Prove that
  \[F_1+F_3+F_5+\dots+F_{2n-1}=F_{2n}.\]
\end{problem*}
\begin{proof}
  We will use induction.\\
  \indent\textit{Base Case:} \(n=1.\) \(F_1=F_2=1.\)\\
  \indent\textit{Induction Hypothesis:} Assume our desired equation is true for all \(n\leq k.\) We will now show that it also holds for \(n=k+1.\) Note that
  \[(F_1+F_3+\dots+F_{2k-1})+F_{2(k+1)-1}=F_{2k}+F_{2k+1}=F_{2k+2}=F_{2(k+1)}.\] Hence we are done.
\end{proof}
\begin{problem*}
  [Andrews, NT, Problem 1-1.17]
  Prove that \(n(n^2-1)(3n+2)\) is divisible by 24 for each positive integer \(n\).
\end{problem*}
\begin{proof}
  We will use induction.\\
  \indent\textit{Base Case:} \(n=1.\) Note that \((1)((1)^2-1)(3(1)+2)=0,\) which is divisible by 24.\\
  \indent\textit{Induction Hypothesis:} Assume the problem statement is true for all \(n\leq k\). We will now show it is true for \(n=k+1.\) Note that\
  \begin{align*}
     & (k+1)((k+1)^2-1)(3(k+1)+2)                            \\
     & =(k+1)(k^2+2k)(3k+5)                                  \\
     & = k(k+1)(k+2)(3k+5)                                   \\
     & = k(k+1)((k-1)+3)((3k+2)+3)                           \\
     & = k(k+1)(k-1)(3k+2)+k(k+1)\cdot[3(3k+2)+3(k-1)+3(3)].
  \end{align*}
  We know the first term of the RHS is divisible by 24 by our Induction Hypothesis. It suffices to show that \(k(k+1)\cdot[3(3k+2)+3(k-1)+3(3)]\) is divisible by 24. It follows that
  \begin{align*}
     & k(k+1)\cdot[3(3k+2)+3(k-1)+3(3)] \\
     & = k(k+1)[9k+6+3k-3+9]            \\
     & = k(k+1)(12k+12)                 \\
     & = 12k(k+1)(k+1).
  \end{align*}
  It is obvious that one of \(k,k+1\) is even. Hence we are done.
\end{proof}
\begin{problem*}
  [Andrews, NT, Problem 1-2.4]
  Prove that each integer may be uniquely represented in the form
  \[n=\sum_{j=0}^{s}c_j 3^j,\]
  where \(c_s\neq0,\) and each \(c_j\) is equal to \(-1,0,\) or \(1\).
\end{problem*}
\begin{proof}
  It is easy to show that every non-zero integer can be uniquely represented in base 3. To covert from base 3 to the representation described in the problem, replace \(2\cdot3^k\) with \(1\cdot3^{k+1}+(-1)\cdot3^k\) until all coefficients are \(-1,0,\) or \(1\).
\end{proof}
\begin{problem*}
  [Andrews, NT, Problem 2-1.4]
  Any set of integers \(J\) that fulfills the following two conditions is called an \textit{integral ideal:}
  \begin{enumerate}
    \ii[(i)] if \(n\) and \(m\) are in \(J,\) then \(n+m\) and \(n-m\) are in \(J\)
    \ii[(ii)] if \(n\) is in \(J\) and \(r\) is an integer, then \(rn\) is in \(J.\)
  \end{enumerate}
  Let \(\mathcal{J}_m\) be the set of all integers that are integral multiples of a particular integer \(m.\) Prove that \(\mathcal{J}_m\) is an integral ideal.
\end{problem*}
\begin{proof}
  Note that criteria (i) is satified since for any two multiples of \(m\) (call them \(am\) and \(bm\)), \(am+bm=(a+b)m\) and \(am-bm=(a-b)m\) are in \(\mathcal{J}_m.\)
  Also note that criteria (ii) is satisfied since for any integer \(r\) and any element of the set \(\mathcal{J}_m\) (call this element \(cm\)), \((rc)m\) is in \(\mathcal{J}_m\).
\end{proof}
\newpage
\begin{problem*}
  [Andrews, NT, Problem 2-1.5]
  Prove that every integral ideal \(J\) is identical with \(\mathcal{J}_m\) for some \(m.\)
\end{problem*}
\begin{proof}
  If \(J\neq\{0\}=\mathcal{J}_0,\) then there exist non-zero integers in \(J,\) and with the right choice of \(r\) it is not hard to see that there must exist positive integers in \(J.\)
  By the well-ordering principle of the natural numbers, there must be a least positive integer in \(J,\) say \(m.\) We will now show that \(J=\mathcal{J}_m.\)
  It is clear that every multiple of \(m\) is also in \(J\) by the definition of integral ideals. Also note that \(m\) is not the multiple of any positive integer less than it since there are no positive integers in \(J\) that are less than \(m.\)
  Finally, no non-multiples of \(m\) can be in the set. Say there exists some element \(k\) in \(J\) such that \(m<k\) and \(k\) is not a multiple of \(m.\) Then by Euclid's Division Lemma we have that \(k=qm+r\) for some integers \(q,r<m.\) However, by the definition of integral ideals, \(qm\) is in \(J\), so \(r\) must be in \(J,\) violating the minimality of \(m.\)
  Hence \(J=\mathcal{J}_m,\) as desired.
\end{proof}
\begin{problem*}
  [Andrews, NT, Problem 2-1.6]
  Prove that if \(a\) and \(b\) are odd integers, then \(a^2-b^2\) is divisble by 8.
\end{problem*}
\begin{proof}
  Let \(a=2k+1\) and \(b=2l+1\) for some integers \(k,l.\) Then
  \[a^2-b^2=(2k+1)^2-(2l+1)^2=4(k^2+k-(l^2+l))=4(k(k+1)-l(l+1)).\]
  Notice that \(k(k+1)\) and \(l(l+1)\) are both even. Hence \(a^2-b^2\) is divisible by 8, as desired.
\end{proof}
\begin{problem*}
  [Andrews, NT, Problem 2-1.7]
  Prove that if \(a\) is an odd integer, then \(a^2+(a+2)^2+(a+4)^2+1\) is divisible by 12.
\end{problem*}
\begin{proof}
  Let \(a=2k+1.\) Then
  \begin{align*}
    a^2+(a+2)^2+(a+4)^2+1 & = (2k+1)^2+(2k+3)^2+(2k+5)^2+1       \\
                          & = 4k^2+4k+1+4k^2+12k+9+4k^2+20k+25+1 \\
                          & = 12k^2+36k+36,
  \end{align*}
  which is obviously divisible by 12.
\end{proof}
\begin{problem*}
  [Andrews, NT, Problem 2-2.4]
  Prove \[\lcm(a,b)=\frac{ab}{\gcd(a,b)}.\]
\end{problem*}
\begin{proof}
  Let \(a=kd\) and \(b=ld\) where \(d=\gcd(a,b).\) Then
  \[\frac{ab}{\gcd(a,b)}=\frac{kd\cdot ld}{d}=kld.\]
  Note that \(k\) and \(l\) are relatively prime. Also note that \(kld\) is a multiple of both \(a\) and \(b\). We have not double-counted any divisors since the greatest common divisor of \(a\) and \(b\) appears only once, hence we are done.
\end{proof}
\begin{problem*}
  [100 Geometry Problems \#20, Sharygin 2014]
  Let \(ABC\) be an isosceles triangle with base \(AB\). Line \(\ell\) touches its circumcircle at point \(B\). Let \(CD\) be a perpendicular from \(C\) to \(\ell\), and \(AE, BF\) be the altitudes of \(ABC\). Prove that \(D,E,F\) are collinear.
\end{problem*}
\begin{proof}
  Denote by \(H\) the orthocenter of \(\triangle ABC\). Also, call the intersection of \(\ol{CD}\) and \(\ol{EH}\) point \(G.\)

  \textbf{Lemma 1:} \(G\) lies on \((ABC)\).

  \begin{subproof}
    Note that \(B,E,G,D\) are concyclic since \(\dang GEB = \dang GDB = 90\degree\). Thus
    \[\dang CGA = \dang CGE = \dang DGE = \dang DBE = \dang DBC = \dang BAC = \dang CBA.\]
    Hence \(A,B,C,G\) are concylic, as desired.
  \end{subproof}

  \textbf{Lemma 2:} \(\triangle CEH \sim \triangle CDB.\)

  \begin{subproof}
    It is well known that the reflection of an orthocenter of a triangle along one of its sides coincides with its circumcircle. Since \(G\) lies on \(\ol{EH}\) and, by Lemma 1, \(G\) lies on \((ABC)\), \(G\) is the reflection of \(H\) along \(CB\), making \(\triangle HCG\) isosceles. Note that \(\dang CHE = \dang EGC = \dang CBD\) and \(\dang CEH = \dang CDB = 90\degree\), so \(\triangle CEH \sim \triangle CDB\) by \(AA\sim\).
  \end{subproof}

  The previous result shows that there exists a spiral similarity between \(\triangle CEH\) and \(\triangle CDB\) centered at \(C\). Note that \(B\) is on \(\ol{HF}\), so it follows that \(D\) is on \(\ol{EF}\). Hence, we are done.
\end{proof}
\begin{problem*}
  [100 Geometry Problems \#21, Purple Comet 2013]
  Two concentric circles have radii 1 and 4. Six congruent circles form a ring where each of the six circles is tangent to the two circles adjacent to it as shown. The three lightly shaded circles are internally tangent to the circle with radius 4 while the three darkly shaded circles are externally tangent to the circle with radius 1. The radius of the six congruent circles can be written \(\frac{k+\sqrt{m}}{n}\), where \(k,m\) and \(n\) are integers with \(k\) and \(n\) relatively prime. Find \(k+m+n\).
\end{problem*}
\begin{soln}
  Number each of the six congruent circles from 1 to 6, with the first circle being lightly colored. We know that the centers of all six congruent circles \((O_1,O_2,\dots,O_6)\) are evenly spaced around the center of the two concentric circles, \(O\), by symmetry. That is, \(\angle O_1OO_2 = 60\degree.\) Denote by \(r\) the radius of each of the six circles. Note that \(O_1O=4-r\), \(O_2O = r+1\), and \(O_1O_2 = 2r\). Then by LoC on \(\triangle O_1OO_2\) we have
  \begin{align*}
    (2r)^2 & = (r+1)^2+(4-r)^2-2(r+1)(4-r)\cos(60\degree) \\
    (2r)^2 & = (r+1)^2+(4-r)^2-(r+1)(4-r)                 \\
    4r^2   & = r^2+2r+1+16-8r+r^2-(3r+4-r^2)              \\
    0      & = r^2 +9r-13.
  \end{align*}
  Using the quadratic formula gives us
  \[r=\frac{-9+\sqrt{133}}{2}.\]
  Thus our final answer is \(-9+133+2 = \fbox{126}.\)
\end{soln}
\begin{problem*}
  [100 Geometry Problems \#22]
  Let \(A,B,C,D\) be points in the plane such that \(\angle BAC = \angle CBD.\) Prove that the circumcircle of \(\triangle ABC\) is tangent to \(BD\).
\end{problem*}
\begin{proof}
  Let \(D\) be on the tangent of \(B,\) and let \(D'\) be on \((ABC)\) while on the same side of \(BC\) as \(D\). Note that \(\dang BAC = \dang BD'C\). As \(D'\) approaches \(B,\) line \(BD'\) approaches a tangent to \((ABC),\) but our angle property does not change. Hence, in the limit, \(\dang BAC = \dang CBD.\)
\end{proof}
\begin{problem*}
  [100 Geometry Problems \#23, Britain 1995]
  Triangle \(ABC\) has a right angle at \(C\). The internal bisectors of angles \(BAC\) and \(ABC\) meet \(BC\) and \(CA\) at \(P\) and \(Q\) respectively. The points \(M\) and \(N\) are the feet of the perpendiculars from \(P\) and \(Q\) to \(AB\). Find angle \(MCN\).
\end{problem*}
\begin{soln}
  Note that \(BCQN\) and \(ACPM\) are cyclic quadrilaterals. Now consider \(\triangle CMN\). We have
  \[\angle CNM = 90\degree - \angle QNC = 90\degree - \angle QBC = 90\degree - \frac12\angle B.\]
  Similarly, we have
  \[\angle CMN = 90\degree - \angle PMC = 90\degree - \angle CAP = 90\degree - \frac12\angle A.\]
  Considering \(\triangle CMN\) as a whole, we have
  \begin{align*}
    \angle MCN & = 180\degree - \angle CNM + \angle CMN                                        \\
               & = 180- \left(90\degree - \frac12\angle B + 90\degree - \frac12\angle A\right) \\
               & = \frac12(\angle A + \angle B)                                                \\
               & = \frac12\angle C                                                             \\
               & = \fbox{\(45\degree\)}.
  \end{align*}
\end{soln}
\begin{problem*}
  [100 Geometry Problems \#24]
  Let \(ABCD\) be a parallelogram with \(\angle A\) obtuse, and let \(M\) and \(N\) be the feet of the perpendicualrs from \(A\) to sides \(BC\) and \(CD\). Prove that \(\triangle MAN \sim \triangle ABC\).
\end{problem*}
\begin{proof}
  Since \(\dang AMC = \dang ANC = 90\degree, AMCN\) is a cyclic quadrilateral. Thus
  \[\dang AMN = \dang ACN = \dang CAB\]
  where the second equality holds because \(AB \parallel CD\). Also note that
  \[\dang BCA = \dang MCA = \dang MNA.\]
  Therefore \(\triangle MAN \sim \triangle ABC\) by \(AA\sim\).
\end{proof}
\begin{problem*}
  [100 Geometry Problems \#25]
  For a given triangle \(\triangle ABC\), let \(H\) denote its orthocenter and \(O\) its circumcenter.
  \begin{enumerate}
    \ii[(a)] Prove that \(\angle HAB = \angle OAC\).
    \ii[(b)] Prove that \(\angle HAO = |\angle B - \angle C|\).
  \end{enumerate}
\end{problem*}
\begin{proof}
  We will prove part (a) first. Note that
  \[\angle BAH = 90\degree - \angle CBA = 90\degree - \frac12 \angle COA = \angle OAC.\]
  (This was Problem 4.23 of Evan Chen's EGMO).\\
  We will now prove part (b). Note that \(\angle BAH = 90\degree - \angle B\) and \(\angle CAH = 90\degree - \angle C\). Finally
  \[\angle HAO = |\angle CAH - \angle BAH| = |90\degree - \angle C - (90 \degree - \angle B)| = |\angle B - \angle C|.\]
  The absolute value symbols are required to account for cases where \(\angle B < \angle C\). Hence, we are done.
\end{proof}
\begin{problem*}
  [100 Geometry Problems \#27, AMC 12A 2012]
  Circle \(C_1\) has its center \(O\) lying on circle \(C_2\). The two circles meet at \(X\) and \(Y\). Point \(Z\) in the exterior of \(C_1\) lies on circle \(C_2\) and \(XZ=13,OZ=11,\) and \(YZ=7\). What is the radius of circle \(C_1\)?
\end{problem*}
\begin{soln}
  For ease of notation denote by \(O_1\) the center of circle \(C_1\) and by \(O_2\) the center of circle \(C_2\). Also denote by \(\theta\) the measure of \(\angle O_1YZ\). Note that \(O_1, X,Y,Z\) all lie on \(C_2\). From this we know that \(\angle O_1XZ = 180\degree - \theta\). Also note that \(O_1X=r=O_1Y,\) where \(r\) is the radius of circle \(C_1\). By LoC on \(\triangle O_1YZ\) and \(\triangle O_1XZ\) we have
  \[
    \begin{cases}
      7^2+r^2-14r\cos\theta=11^2 \\
      13^2+r^2-26r\cos(180\degree-\theta)=11^2.
    \end{cases}
  \]
  After noticing that \(\cos(180\degree-\theta)=-\cos\theta\), we can solve for \(r\cos\theta\) and set the two equations equal to each other, leaving us with
  \[\frac{r^2+7^2-11^2}{14}=\frac{r^2+13^2-11^2}{-26}.\]
  Solving for \(r\) gives us a final answer of \fbox{\(\sqrt{30}\)}.
\end{soln}
\begin{problem*}
  [100 Geometry Problems \#28]
  Let \(ABCD\) be a cyclic quadrilateral with no two sides parallel. Lines \(AD\) and \(BC\) (extended) meet at \(K\), and \(AB\) and \(CD\) (extended) meet at \(M\). The angle bisector of \(\angle DKC\) intersects \(CD\) and \(AB\) at points \(E\) and \(F\), respectively; the angle bisector of \(\angle CMB\) intersects \(BC\) and \(AD\) at points \(G\) and \(H\), respectively. Prove that quadrilateral \(EGFH\) is a rhombus.
\end{problem*}
\begin{proof}
  Let \(\alpha = \angle AKF = \angle FKB\) and \(\beta = \angle BMG = \angle GMC\). Call the intersection of \(EF\) and \(GH\) point \(X\). Note that \(\angle AFK = 180\degree - (A + \alpha)\), so \(\angle KFB = A + \alpha\), then \(\angle MXF = 180\degree - (A + \alpha + \beta)\). Now note that since \(180\degree - C = A\), \(\angle KEC = 180\degree - (\alpha + A)\), so \(\angle XEC = A + \alpha \). Then \(\angle MXE = 180\degree - (A + \alpha + \beta) = \angle MXF,\) and since \(E,X,F\) are collinear, \(\angle MXE = \angle MXF = 90\degree\). Hence the diagonals of quadrilateral \(EGFH\) are perpendicular. Also note that \(\triangle MXF \cong \triangle MXE\) by \(ASA\cong\), so it follows that \(XF = XE\). A very similar process can be done to determine that \(XG = XH\), hence \(EGFH\) is a rhombus, and we are done.
\end{proof}
\begin{problem*}
  [100 Geometry Problems \#29]
  In \(\triangle ABC, AB = 13, AC = 14,\) and \(BC = 15\). Let \(M\) denote the midpoint of \(\ol{AC}\). Point \(P\) is placed on line segment \(\ol{BM}\) such that \(\ol{AP} \perp \ol{PC}\). Suppose that \(p, q,\) and \(r\) are positive integers with \(p\) and \(r\) relatively prime and \(q\) squarefree such that the area of \(\triangle APC\) can be written in the form \(\frac{p\sqrt{q}}{r}.\) What is \(p + q + r\)?
\end{problem*}
\begin{soln}
  Denote by \(D\) the foot of the altitude from \(B\) to \(AC\). With some simple calculations we find that \(AD = 5\), \(BD = 12\), and \(MD = 2\). Then by the Pythagorean Theorem we find that \(BM = 2\sqrt{37}\). Denote by \(E\) the foot of the altitude from \(P\) to \(AC\). Then we know that \(\triangle PME \sim \triangle BMD\) by \(AA\sim\). So,
  \[\frac{PM}{PE}=\frac{BM}{BD}\Longrightarrow\frac{7}{PE}=\frac{2\sqrt{37}}{12}\Longrightarrow PE = \frac{42}{\sqrt{37}}.\]
  Note that \(PE\) is the height of \(\triangle APC\). Thus
  \[[APC]= \frac12\cdot\frac{42}{\sqrt{37}}\cdot14= \frac{294\sqrt{37}}{37},\]
  so our final answer is \(294+37+37=\fbox{368}\).
\end{soln}
\begin{problem*}
  [100 Geometry Problems \#30]
  Acute-angled triangle \(ABC\) is inscribed into circle \(\Omega\). Lines tangent to \(\Omega\) at \(B\) and \(C\) intersect at \(P\). Points \(D\) and \(E\) are on \(AB\) and \(AC\) such that \(PD\) and \(PE\) are perpendicular to \(AB\) and \(AC\) respectively. Prove that the orthocenter of triangle \(ADE\) is the midpoint of \(BC\).
\end{problem*}
\begin{proof}
  Denote by \(M\) the midpoint of \(BC\). Also denote by \(D'\) and \(E'\) the intersection of \(\ol{MD}\) with \(AC\) and \(\ol{ME}\) with \(AB\), respectively. Note that \(BC \perp MP\). Then since \(\dang BDP = 90\degree = \dang BMP,\) \(BDPM\) is a cyclic quadrilateral. Then
  \[\dang D'DA = \dang MDB = \dang MPB = 90\degree - \dang PBM = 90\degree - \dang PBC = 90\degree - \dang BAC,\]
  hence \(\dang AD'D = 90\degree\). Similarly, \(\dang EE'A = 90\degree\), so we are done.
\end{proof}
\begin{problem*}
  [100 Geometry Problems \#31]
  For an acute triangle \(\triangle ABC\) with orthocenter \(H\), let \(H_A\) be the foot of the altitude from \(A\) to \(BC\), and define \(H_B\) and \(H_C\) similarly. Show that \(H\) is the incenter of \(\triangle H_AH_BH_C\).
\end{problem*}
\begin{proof}
  Note that \(HH_ABH_C\) and \(HH_BCH_A\) are cyclic quadrilaterals. Then by angle chasing we get
  \[\dang HH_AH_C = \dang HBH_C = \dang H_BBA = 90\degree - \dang BAC\]
  and
  \[\dang H_BH_AH = \dang H_BCH = \dang ACH_C = 90\degree - \dang BAC.\]
  Thus, \(\dang HH_AH_C = \dang H_BH_AH,\) so \(H\) lies on the angle bisector of \(\dang H_BH_AH_C\). Similarly, \(H\) lies on the angle bisectors of \(\dang H_AH_CH_B\) and \(\dang H_CH_BH_A\), so \(H\) is the incenter of \(\triangle H_AH_BH_C\), as desired.
\end{proof}
\newpage
\begin{problem*}
  [100 Geometry Problems \#32, AMC 10A 2013] In \(\triangle ABC, AB = 86\), and \(AC = 97\). A circle with center \(A\) and radius \(AB\) intersects \(\ol{BC}\) at points \(B\) and \(X\). Moreover \(\ol{BX}\) and \(\ol{CX}\) have integer lengths. What is \(BC\)?
\end{problem*}
\begin{soln}
  Let \(BX = x\) and \(CX = y\). By Power of a Point on \(C\), we have 
  \[(97-86)(97+86) = 11\cdot183 = 3\cdot11\cdot61=y(x+y).\]
  Note that \(y,x+y\) are both positive integers, and \(y<x+y\). If \((y,x+y) = (1,3\cdot11\cdot61),(3,11\cdot61),\) or \((11,3\cdot61)\), the triangle inequality on \(\triangle ACX\) would be violated. Therefore \(BC = x+y = \fbox{61}\), our only remaining possiblity.
\end{soln}
\begin{problem*}
  [The CALT Induction Handout Exercise 2.3, NICE Spring 2021]
  Fifty rooms of a castle are lined in a row. The first room contains 100 knights, while the remaining 49 rooms contain one knight each. These knights wish to escape the castle by breaking the barriers between consecutive rooms, ending with the barrier from room 50 to the outside. At the stroke of midnight, each knight in the \(i\)-th room begins breaking the barrier between the \(i\)-th and \((i + 1)\)-st rooms, where we count the 51st room as the exterior. Each person works at a constant rate and is able to break down a barrier in 1 hour, and once a group of knights breaks down the \(i\)-th barrier, they immediately join the knight breaking down the \((i + 1)\)-st barrier. The number of hours it takes for the knights to escape the castle is \(\tfrac{m}{n}\), where \(m\) and \(n\) are positive relatively prime integers. Compute the product \(mn\).
\end{problem*}
\begin{soln}
  Denote by ``room 0'' the first room and every remaining room analagously. Denote by \(t_n\) the time it takes to break down the wall in room \(n\). Note that 
  \begin{align*}
    t_0 &= \frac{1}{100} \\
    t_1 &= t_0 + (1-t_0)\cdot\frac{1}{100+1} = \frac{100}{101}t_0+\frac{1}{101} = \frac{2}{101}\\
    t_2 &= t_1 + (1-t_1)\cdot\frac{1}{100+2} = \frac{101}{102}t_1+\frac{1}{102} = \frac{3}{102}\\
    t_3 &= t_2 + (1-t_2)\cdot\frac{1}{100+3} = \frac{102}{103}t_2+\frac{1}{103} = \frac{4}{103}\\
    &\vdots \\
    t_i &= \underbrace{t_{i-1}}_{\substack{\text{current wall already}\\\text{done by } i\text{-th person}}} + \underbrace{(1-t_{i-1})\cdot\frac{1}{100+i}}_{\substack{\text{rest of wall with}\\100 + i \text{ people}}} = \frac{i+1}{100+i}. 
  \end{align*}
  Thus, \(t_{49} = \frac{50}{149}\), so our final answer is \(50\cdot149 = \fbox{7450}\).
\end{soln}
\begin{problem*}
  [CALT Induction Exercise 2.4, PUMaC 2018]
  If \(a_1,a_2,\dots\) is a sequence of real numbers such that for all \(n\), \[\sum_{k=1}^n a_k\left(\frac{k}{n}\right)^2=1,\] find the smallest \(n\) such that \(a_n < \frac{1}{2018}\).
\end{problem*}
\begin{soln}
  We will first begin by finding the first few values of \(a_k\). By some simple calculations it follows that
  \begin{align*}
    a_1\cdot\left(\tfrac{1}{1}\right)^2&= 1 \Longrightarrow a_1 = \tfrac{1}{1}\\
    a_1\cdot\left(\tfrac{1}{2}\right)^2+a_2\cdot\left(\tfrac{2}{2}\right)^2&= 1 \Longrightarrow a_2 = \tfrac{3}{4}\\
    a_1\cdot\left(\tfrac{1}{3}\right)^2+a_2\cdot\left(\tfrac{2}{3}\right)^2+a_3\cdot\left(\tfrac{3}{3}\right)^2 &= 1 \Longrightarrow a_3 = \tfrac{5}{9}\\
    a_1\cdot\left(\tfrac{1}{4}\right)^2+a_2\cdot\left(\tfrac{2}{4}\right)^2+a_3\cdot\left(\tfrac{3}{4}\right)^2 + a_4\cdot\left(\tfrac{4}{4}\right)^2&= 1 \Longrightarrow a_4 = \tfrac{7}{16}.
  \end{align*}
  We can now conjecture that
  \[a_k = \frac{2k-1}{k^2}.\]
  Now we need to find an \(n\) such that
  \[\frac{2n-1}{n^2}<\frac{1}{2018}.\]
  Reciprocating both sides and doing polynomial long division on the LHS gives us
  \[\frac{n^2}{2n-1} = \frac{1}{2}n+\frac{1}{4}+\frac{\tfrac14}{2n-1}>2018.\]
  Multiplying both sides by 4 gives us 
  \[2n+1+\frac{1}{2n-1}>4\cdot2018.\]
  Note that \(n=2\cdot2018\) satsifies this inequality while \(n=2\cdot2018-1\) does not. Hence our answer is \(2\cdot2018 = \fbox{4036}\).
\end{soln}
\begin{problem*}
  [2016 AMC 12B \#21]
  Let $ABCD$ be a unit square. Let $Q_1$ be the midpoint of $\overline{CD}$. For $i=1,2,\dots,$ let $P_i$ be the intersection of $\overline{AQ_i}$ and $\overline{BD}$, and let $Q_{i+1}$ be the foot of the perpendicular from $P_i$ to $\overline{CD}$. What is\[\sum_{i=1}^{\infty} \text{Area of } \triangle DQ_i P_i \, ?\]
\end{problem*}

% I cannot get this to work. VS Code gives no feedback for .asy files, either. Good luck future me!
\begin{xcomment}{}
\begin{figure}[h]
  \centering
  \begin{asy}
    import graph;
    import olympiad;
    import cse5;
    defaultpen(fontsize(10pt));
    usepackage("amsmath");
    usepackage("amssymb");
    settings.tex="latex";
    settings.outformat="pdf";
    size(7cm);

    pair A = (0,1);
    pair B = (1,1);
    pair C = (1,0);
    pair D = (0,0);

    draw(A--B--C--D--cycle);
    draw(B--D);
    pair Q1 = midpoint(C--D);
    draw(A--Q1);
    pair P1 = IP(B--D,A--Q1);

    dot("$A = (0,1)$", A, dir(A));
    dot("$B = (1,1)$", B, dir(B));
    dot("$C = (1,0)$", C, dir(270));
    dot("$D = (0,0)$", D, dir(225));
    dot("$Q_1 = (\tfrac12,0)$", Q1, dir(270));
    dot("$P_1$", P1, dir(0));
  \end{asy}
  \caption{2016 AMC 12B \#21}
\end{figure}
\end{xcomment}

\begin{soln}
We will use coordbash. First, let \(D=(0,0),A=(0,1),B=(1,1), \text{and }C=(1,0)\). Note that \(\ol{AQ_1}\) is equivalent to \(y=-2x+1\) and \(\ol{BD}\) is equivalent to \(y=x\) in this setup. Solving this system gives us \(P_1 = (\tfrac13,\tfrac13)\). Since \(Q_2\) is the foot of the altitude from \(P_1\) to \(\ol{CD}\), the \(x\)-coordinate of \(Q_2\) is the same as for \(P_1\) while the \(y\)-coordinate is just 0. In other words, \(Q_2 = (\tfrac13,0)\). We now see that \(\ol{AQ_2}=(y=-3x+1)\), \(P_2 = (\tfrac14,\tfrac14)\), and \(Q_3 = (\tfrac14,0)\). In general, we have
\begin{align*}
  P_i &= \left(\frac{1}{i+2},\frac{1}{i+2}\right)\\
  Q_i &= \left(\frac{1}{i+1},0\right).
\end{align*}

In general, to find \([DQ_iP_i]\), note that \(DQ_i\) is the base of the triangle and \(P_iQ_{i+1}\) is the height. Thus
\[[DQ_iP_i] = \frac{1}{2}\cdot\frac{1}{i+1}\cdot\frac{1}{i+2}.\]
What we have left to evaluate is the following sum
\[\frac12\sum_{i=1}^{\infty}\frac{1}{i+1}\cdot\frac{1}{i+2}.\]
Using partial fraction decomposition on the summand gets us
\[\frac12\left[\sum_{i=1}^{\infty}\frac{1}{i+1}-\sum_{i=1}^{\infty}\frac{1}{i+2}\right].\]
Note that we have a telescoping sum. It is not hard to see by writing out a few terms that our sum is equal to \(\fbox{\(\frac14\)}\).
\end{soln}

\begin{problem*}
  [Christmas Math Competitions 12B 2021 \#15]
  There are $n$ values of $x$ which satisfy
  \[\lfloor x \rfloor ^2 = \{x\}^2 + \dfrac{2020}{2021}x^2.\]
  What is the remainder when $n$ is divided by $5$? (Here, $\lfloor\bullet\rfloor$ is the greatest integer function and $\{\bullet\}$ is the fractional part function.)
\end{problem*}
\begin{soln}
  Let \(x=A+\varepsilon\) where \(A\) is the integer part of \(x\) and \(\varepsilon\) is the fractional part. Substituting this into our given equation gives us
  \[A^2 = \varepsilon^2 + \frac{2020}{2021}(A+\varepsilon)^2 = \varepsilon^2 + \frac{2020}{2021}(A^2+2A\varepsilon+\varepsilon^2).\]
  Distributing \(\tfrac{2020}{2021}\) and moving all the terms to the LHS gives us
  \begin{align*}
    \frac{1}{2021}A^2-\frac{4040}{2021}A\varepsilon - \frac{4041}{2021}\varepsilon^2 &= 0\\
    A^2-4040A\varepsilon-4041\varepsilon^2 &= 0 \\
    (A-4041\varepsilon)(A+\varepsilon) &= 0.
  \end{align*}
  Thus we have two cases. 
  \begin{itemize}
    \ii  For the case where \(A+\varepsilon=0,\) since \(A\) is an integer, the only possible value for \(\varepsilon\) is 0, which then implies that \(A=0,\) giving us one value of \(x\) in this case.
    \ii For the case where \(A-4041\varepsilon =0,\) for the LHS to be an integer, the denominator of \(\varepsilon\) must be 4041. Thus \(\varepsilon = \tfrac{1}{4041},\tfrac{2}{4041},\dots,\tfrac{4040}{4041},\) giving us 4040 values of \(x\) in this case.
  \end{itemize}
  Finally, note that \(1+4040 \equiv \fbox{1} \pmod{5}.\)
\end{soln}
\end{document}