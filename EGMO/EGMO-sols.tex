\documentclass[letterpaper,oneside]{book}

\usepackage{amsmath}
\usepackage{amsthm}
\usepackage{amssymb}
\usepackage{textcomp}
\usepackage{gensymb}

\usepackage{tikz}
\usepackage{pgfplots}
\pgfplotsset{compat=1.15}
\usepackage{mathrsfs}
\usetikzlibrary{arrows}

\usepackage{asymptote}

\providecommand{\ol}{\overline}
\providecommand{\ul}{\underline}
\providecommand{\wt}{\widetilde}
\providecommand{\wh}{\widehat}
\providecommand{\eps}{\varepsilon}
\providecommand{\half}{\frac{1}{2}}
\providecommand{\inv}{^{-1}}
\newcommand{\dang}{\measuredangle} %% Directed angle
\providecommand{\CC}{\mathbb C}
\providecommand{\FF}{\mathbb F}
\providecommand{\NN}{\mathbb N}
\providecommand{\QQ}{\mathbb Q}
\providecommand{\RR}{\mathbb R}
\providecommand{\ZZ}{\mathbb Z}
\providecommand{\ts}{\textsuperscript}
\providecommand{\dg}{^\circ}
\providecommand{\ii}{\item}
\providecommand{\defeq}{\coloneqq}
\DeclareMathOperator*{\lcm}{lcm}
\DeclareMathOperator*{\argmin}{arg min}
\DeclareMathOperator*{\argmax}{arg max}
\providecommand{\hrulebar}{\par
\hspace{\fill}\rule{0.95\linewidth}{.7pt}\hspace{\fill}
\par\nointerlineskip \vspace{\baselineskip}}
\setlength{\parindent}{0pt}

\begin{document}
\title{Euclidean Geometry in Mathematical Olympiads Solutions}
\author{Ryder Pham}
\maketitle

\chapter{}
\textbf{Problem 1.51 (IMO 1985/1)}. A circle has center on the side $\ol{AB}$ of the cyclic quadrilateral $ABCD$. The other three sides are tangent to the circle. Prove that $AD + BC = AB$.
\begin{proof}
  Call the center of the circle point $O.$ Let the point $T$ be where the circumcircle of $CDO$ intersects $\ol{AB}.$ By arc measures
\end{proof}
\chapter{}
\textbf{Lemma 2.19}. Prove that the A-exradius has length
$$r_a = \frac{s}{s-a}r,$$
where $r$ is the inradius.

\begin{proof}
    Drop perpendiculars from $I$ and $I_A$ to $AB$. Call the feet of these perpendiculars $B_1$ and $B_2$ respectively. Notice that $IB_1 = r$ and $I_AB_2 = r_a$ and that $\triangle AB_1I \sim \triangle AB_2I_A$. Therefore 

$$\frac{r}{r_a} = \frac{AB_1}{AB_2},$$

but by Lemmas 2.15 and 2.17, we know that $AB_1 = s-a$ and $AB_2 = s$, hence 

$$r_a = \frac{s}{s-a}r,$$

and we are done.
\end{proof}

\textbf{Lemma 2.20}.   Let $ABC$ be a triangle. Suppose its incircle and $A$-excircle are tangent to $BC$ at $D$ and $X$, respectively. Show that $BD = CX$ and $BX = CD$. 

\begin{proof}  We will first show that $BD=CX$. Let the incircle be tangent to side $AB$ at point $F$ and let to side $AC$ at point $E$. Let the $A$-excircle be tangent to the extension of line $AC$ at $C_1$ and to the extension of line $AB$ at $B_1$. Then

\begin{align*}
BD &= BF \\
&= AB_1 - AF - BB_1 \\
&= (AC_1 - AE) - BX \\
&= (CC_1 + CE) - (BC - CX) \\
&= CX + (CD - BC) + CX \\
&= 2CX - BD \\
2BD&=2CX \rightarrow BD = CX.
\end{align*}

It follows that $BX=CD$ because

\begin{align*}
BD &= CX \\
BD + DX &= DX + CX \\
BX &= CD. \space 
\end{align*}
\end{proof}
  \textbf{Lemma 2.24}.   Let $ABC$ be a triangle with $I_A, I_B,$ and $I_C$ as excenters. Prove that triangle $I_AI_BI_C$ has orthocenter $I$ and that triangle $ABC$ is its orthic triangle.

  \begin{proof}  By the Incenter-Excenter Lemma, we know that $AI_A, BI_B,$ and $CI_C$ coincide at the incenter $I$. We also know from the Lemma that $II_A$ is the diameter of circle $BICI_A.$ Therefore we have that 

$$\angle {I_CCI_A} = \angle {ICI_A} =  90\degree \text{ and } \angle {I_BBI_A} = \angle {IBI_A} = 90\degree .$$

This follows similarily for $II_B$ and $II_C.$ Now we know that $AI_A, BI_B, CI_C$ are in fact the altitudes of triangle $I_AI_BI_C,$ therefore $I$ is the orthocenter of triangle $I_AI_BI_C.$ Note that since $A,B,$ and $C$ are the feet of the altitudes, $ABC$ is the orthic triangle of triangle $I_AI_BI_C.$ \end{proof}

  \textbf{Theorem 2.25 (The Pitot Theorem)}.   Let $ABCD$ be a quadrilateral. If a circle can be inscribed in it, prove that $AB + CD = BC + DA.$

  \begin{proof}  Call the points where $AB, BC, CD, DA$ are tangent to the circle $E, F, G, H,$ respectively. Let $AE = AH = a, BE = BF = b, CF = CG = c, DG = DH = d.$ Now note that our condition can be manipulated as follows:

\begin{align*}
    AB + CD &= BC + DA \\
    (AE + BE) + (CG + DG) &= (BF + CF) + (AH + DH)\\
    a + b + c + d &= b + c + a + d.
\end{align*}

Hence, we are done. \end{proof}

  \textbf{Problem 2.26 (USAMO 1990/5)}.   An acute-angled triangle $ABC$ is given in the plane. The circle with diameter $\overline{AB}$ intersects altitude $\overline{CC'}$ and its extension at points $M$ and $N$, and the circle with diameter $\overline{AC}$ intersects altitude $\overline{BB'}$ and its extensions at $P$ and $Q$. Prove that the points $M, N, P , Q$ lie on a common circle.

  \begin{proof}  Let the circle with diameter $\overline{AB}$ be called $\omega_1$ and the circle with diameter $\overline{AC}$ be called $\omega_2.$  By Theorem 2.9, it suffices to show that the intersection of $\overline{MN}$ and $\overline{PQ}$ lies on the radical axis of $\omega_1$ and $\omega_2.$ Since $\overline{MN}$ and $\overline{PQ}$ are altitudes of $\triangle ABC,$ their intersection is the orthocenter of $\triangle ABC.$ We will call this point $H.$ Note that $\overline{AH}$ is the third altitude of $\triangle ABC.$ Call the foot of this altitude $A'.$ Now note that $\angle AA'B = \angle AA'C = 90\degree,$ and since $\overline{AB}$ and $\overline{AC}$ are diameters of their respective circles, $\omega_1$ and $\omega_2$ must intersect at $A'.$ Hence, $\overline{AA'}$ is the radical axis of $\omega_1$ and $\omega_2,$ and since $A,A',H$ are colinear, $H$ lies on this line. \end{proof}

  \begin{figure}[h]
    \centering
    \begin{asy}
      import graph;
import olympiad;
import cse5;
defaultpen(fontsize(10pt));
usepackage("amsmath");
usepackage("amssymb");
settings.tex="latex";
settings.outformat="pdf";
size(9cm);

pair A = dir(120);
pair B = dir(210);
pair C = dir(330);
pair Cprime = foot(C,A,B);
pair Bprime = foot(B,A,C);
path AB = A--B;
path AC = A--C;
path cAB = Circle(midpoint(AB),distance(midpoint      (AB),A));
path cAC = Circle(midpoint(AC),distance(midpoint      (AC),A));
draw(A--B--C--cycle);
draw(cAB);
draw(cAC);
path CCprime =  L(C, Cprime);
path BBprime =  L(B, Bprime);
pair []  m = IPs(CCprime, cAB);
pair []  p = IPs(cAC, BBprime);
pair M = m[1];
pair N = m[0];
pair P = p[0];
pair Q = p[1];
draw(M--C);
draw(P--B);
draw(circumcircle(M,N,P), dashed);

pair [] H = IPs(CCprime, BBprime);
pair H = H[0];
pair Aprime = foot(A,B,C);
draw(A--Aprime);
draw(rightanglemark(B, Cprime, C, 3));
draw(rightanglemark(B, Bprime, A, 3));
draw(rightanglemark(A, Aprime, C, 3));

dot("$A$", A, 1.5dir(90));
dot("$B$", B, dir(B));
dot("$C$", C, dir(C));
dot("$A'$", Aprime, 1.5dir(-100));
dot("$B'$", Bprime, 2dir(0));
dot("$C'$", Cprime, dir(30));
dot("$H$", H, 2dir(20));
dot("$M$", M, dir(M));
dot("$N$", N, dir(-90));
dot("$P$", P, dir(P));
dot("$Q$", Q, 2dir(-100));
    \end{asy}
    \caption{Problem 2.26}
  \end{figure}

  \textbf{Problem 2.27 (BAMO 2012/4)}.   Given a segment $\overline{AB}$ in the plane, choose on it a point $M$ different from $A$ and $B$. Two equilateral triangles $AMC$ and $BMD$ in the plane are constructed on the same side of segment $\overline{AB}.$ The circumcircles of the two triangles intersect in point $M$ and another point $N.$

(a) Prove that $\overline{AD}$ and $\overline{BC}$ pass through point $N.$

(b) Prove that no matter where one chooses point $M$ along segment $\overline{AB},$ all lines $MN$ will pass through some fixed point $K$ in the plane.

  \begin{proof}  We will prove (a) by angle chasing. Notice that since $ACNM$ and $BDNM$ are cyclic, we have that

$$\angle AMC = \angle ANC = \angle ACM = \angle ANM = \angle MDB = \angle MNB = 60\degree,$$

and since $\angle ANC + \angle ANM + \angle MNB = 60\degree + 60\degree + 60\degree = 180\degree,$ we have that $BC$ is a straight line passing through $N.$ A very similar argument follows for $AD.$

We will now prove (b) using radical axes. First, construct an equilateral triangle $ABE$ on the same side as the other two equilateral triangles. Let the circumcircles around triangles $AMC, BMD,$ and $ABE$ be $\omega_1, \omega_2,$ and $\omega_3,$ respectively. Note that $MN$ is the radical axis of circles $\omega_1$ and $\omega_2,$ the line tangent to circles $\omega_1$ and $\omega_3$ at point $A$ is the radical axis of circles $\omega_1$ and $\omega_3,$ and the line tangent to circles $\omega_2$ and $\omega_3$ at point $B$ is the radical axis of circles $\omega_2$ and $\omega_3.$ Since the centers of  $\omega_1, \omega_2,$ and $\omega_3$ are not colinear, their radical axes (one of which is $MN$) must coincide at the radical center $K.$ Since changing the location of $M$ on $AB$ does not change the tangents at $A$ and $B,$ the point $K$ does not move, hence all possible lines $MN$ must pass through $K.$ \end{proof}

  

  \textbf{Problem 2.28 (JMO 2012/1)}.   Given a triangle $ABC$, let $P$ and $Q$ be points on segments $\overline{AB}$ and $\overline{AC}$, respectively, such that $AP = AQ.$ Let $S$ and $R$ be distinct points on segment $\overline{BC}$ such that $S$ lies between $B$ and $R$, $\angle BPS = \angle PRS$, and $\angle CQR = \angle QSR$. Prove that $P, Q, R, S$ are concyclic.

  \begin{proof}  Since $\angle BPS = \angle PRS$ by the Tangent Criterion, $\overline{AB}$ is tangent to $(PRS).$ Likewise we have that $\overline{AC}$ is tangent to $(QRS).$ Suppose $(PRS)$ and $(QRS)$ are not the same circle. Then since $AP=AQ$ are both tangents to their respective circles, $A$ must lie on the radical axis $\overline{BC},$ but since $ABC$ is a triangle, this is obviously impossible. Hence $P,Q,R,S$ are concyclic. \end{proof}

  

  \textbf{Problem 2.29 (IMO 2008/1)}.   Let $H$ be the orthocenter of an acute-angled triangle $ABC.$ The circle $\Gamma_A$ centered at the midpoint of $\overline{BC}$ and passing through $H$ intersects the sideline $BC$ at points $A_1$ and $A_2.$ Similarly, define the points $B_1, B_2, C_1,$ and $C_2.$ Prove that six points $A_1, A_2, B_1, B_2, C_1,$ and $C_2$ are concyclic.

  \begin{proof}  We will first show that $B_1, B_2, C_1, C_2$ are concyclic. Since $\Gamma_A,\Gamma_B, \Gamma_C$ all intersect at $H$, $H$ is the radical center. We claim that $\overline{AH}$ is the radical axis of $\Gamma_B$ and $\Gamma_C.$ By similar triangles, $M_BM_C$ is parallel to $BC,$ and since $\overline{AH} \perp BC,$ $\overline{AH} \perp M_BM_C.$ The centers of circles $\Gamma_B$ and $\Gamma_C$ are $M_B$ and $M_C,$ respectively, thus $\overline{AH}$ is the radical axis of circles $\Gamma_B$ and $\Gamma_C.$ Since $\overline{B_1B_2}$ and $\overline{C_1C_2}$ intersect at $A,$ by Theorem 2.9 we have shown that $B_1, B_2,C_1,C_2$ are concyclic. Note that the circumcenter of $(B_1B_2C_1C_2)$ is the intersection of the perpendicular bisectors of $B_1B_2$ and $C_1C_2,$ which is the orthocenter $O$ of triangle $ABC.$ Thus what we have proven is that $OB_1 = OB_2 = OC_1 = OC_2.$ A similar argument can be persued for $OA_1$ and $OA_2,$ hence we are done. \end{proof}

  

  \textbf{Problem 2.30 (USAMO 1997/2)}.   Let $ABC$ be a triangle. Take points $D, E, F$ on the perpendicular bisectors of $\overline{BC}, \overline{CA}, \overline{AB}$ respectively. Show that the lines through $A, B, C$  perpendicular to $\overline{EF}, \overline{FD}, \overline{DE}$ respectively are concurrent.

  \begin{proof}  Consider the circles with centers $D,E,F$ with chords $BC, CA, AB,$ respectively. Note that the radical axes of these three circles are the lines through $A,B,C$ perpendicular to $\overline{EF}, \overline{FD}, \overline{DE},$ and since the centers of these three circles are not colinear, their radical axes must intersect at a point. \end{proof} (These centers  can  be colinear, but we won't talk about that)

  

  \textbf{Problem 2.31 (IMO 1995/1)}.   Let  $A, B, C, D$ be four distinct points on a line, in that order. The circles with diameters $\overline{AC}$ and $\overline{BD}$ intersect at $X$ and $Y.$ The line $XY$ meets $\overline{BC}$ at $Z.$ Let $P$ be a point on the line $XY$ other than $Z$. The line $CP$ intersects the circle with diameter $AC$ at $C$ and $M$, and the line $BP$ intersects the circle with diameter $BD$ at $B$ and $N$. Prove that the lines $AM, DN, XY$ are concurrent.

  \begin{proof}  Since $P$ lies on the radical axis of these two circles, and $\overline{BN} \cap \overline{CM} =P,$ $MNBC$ is cyclic by Theorem 2.9. (Reminder that the symbol $\measuredangle$ denotes the directed angle.) Note that

$$\measuredangle NMC = \measuredangle NBC = \measuredangle NBD = 90\degree - \measuredangle BDN =  90\degree - \measuredangle ADN,$$

so 

$$\measuredangle NMA =   \measuredangle NMC -90\degree = (90\degree - \measuredangle ADN) - 90\degree = -\measuredangle ADN = \measuredangle NDA,$$

therefore quadrilateral $DAMN$ is cyclic. The radical axes of the circles $(DAMN), (AMC),$ and $(BND)$ are $\overline{AM}, \overline{DN}, \overline{XY},$ and since the centers of these circles are never colinear, they must intersect at the radical center. \end{proof}

  

  \textbf{Problem 2.32 (USAMO 1998/2)}.   Let $\mathcal{C}_1$ and $\mathcal{C}_2$ be concentric circles, with $\mathcal{C}_2$ in the interior of $\mathcal{C}_1$. From a point $A$ on $\mathcal{C}_1$ one draws the tangent $\overline{AB}$ to $\mathcal{C}_2$  ($B \in  \mathcal{C}_2$). Let $C$ be the second point of intersection of ray $AB$ and $\mathcal{C}_1$, and let $D$ be the midpoint of $\overline{AB}$. A line passing through $A$ intersects $\mathcal{C}_2$ at $E$ and $F$ in such a way that the perpendicular bisectors of $DE$ and $CF$ intersect at a point $M$ on $AB$. Find, with proof, the ratio $AM/MC$.
 
  \begin{proof}  \textbf{INCOMPLETE}
    Note that $CDEF$ is cyclic (need to prove). $M$ is the center of circle $(CDEF).$ Thus $CM=DM.$ \end{proof}

\chapter{}
  \textbf{Theorem 3.2 (Angle Bisector Theorem)}.   Let $ABC$ be a triangle and $D$ a point on $\overline{BC}$ so that $\overline{AD}$ is the internal angle bisector of $\angle BAC.$ Show that

$$\frac{AB}{AC} = \frac{DB}{DC}.$$

 \begin{proof}  Let $\angle BAD = \alpha = \angle CAD$ and $\angle ADB = \beta.$ Note that $\angle ADC = 180\degree - \beta.$ By Law of Sines, we have

$$\frac{DB}{\sin\alpha} = \frac{AB}{\sin\beta} \text{  and  } \frac{DC}{\sin\alpha} = \frac{AC}{\sin(180\degree - \beta)}.$$

Note that $\sin(180\degree-\beta) = \sin\beta.$ Rearranging terms, we have that

$$\frac{\sin\beta}{\sin\alpha} = \frac{AB}{BD} = \frac{AC}{CD}.$$

It follows that $\frac{AB}{AC} = \frac{DB}{DC}.$
 \end{proof}



  \textbf{Problem 3.5}.   Show the trigonometric form of Ceva holds.

 \begin{proof}  Recall that the trigonometric from of Ceva's Theorem is as follows:

Let $\overline{AX}, \overline{BY}, \overline{CZ}$ be cevians of a triangle $ABC.$ They concur if and only if 

$$\frac{\sin \angle BAX \sin \angle CBY \sin \angle ACZ}{\sin \angle XAC \sin \angle YBA \sin \angle ZCB} = 1.$$

By the Law of Sines, we have that

$$\frac{\sin\angle BAX}{BX} = \frac{\sin B}{AX}$$

and

$$\frac{\sin\angle XAC}{XC} = \frac{\sin C}{AX}.$$

Combining these two equations gives us

$$AX = \frac{BX \sin B}{\sin\angle BAX}=\frac{XC \sin C}{\sin\angle XAC} \Rightarrow \frac{\sin \angle BAX}{\sin \angle XAC} = \frac{BX}{XC} \cdot \frac{\sin C}{\sin B}.$$

Similarly, we have that 

$$\frac{\sin \angle CBY}{\sin \angle YBA} = \frac{CY}{YA}\cdot \frac{\sin A}{\sin C} $$

and

$$
\frac{\sin \angle ACZ}{\sin \angle ZCB} = \frac{AZ}{ZB}\cdot \frac{\sin B}{\sin A}. $$

Plugging these values into the original equation, we have that

$$\frac{BX}{XC}\cdot \frac{CY}{YA} \cdot \frac{AZ}{ZB} = 1,$$

and we know this is true from the original statement of Ceva's Theorem.
\end{proof}



 \textbf{ Problem 3.6}.   Let $\overline{AM}, \overline{BE},$ and $\overline{CF}$ be concurrent cevians of a triangle $ABC.$ Show that $\overline{EF} \parallel \overline{BC}$ if and only if $BM = MC.$

 \begin{proof}
  Suppose $\overline{EF} \parallel \overline{BC}.$ Call the point where $\overline{AM}$ intersects $\overline{EF}$ point $Q.$ Notice that $\triangle BPM \sim \triangle EPQ$ and $\triangle CPM \sim \triangle FPQ.$ Thus we have the following relationship:

$$\frac{BM}{EQ} = \frac{MP}{QP} = \frac{CM}{FQ}.$$

Now also notice that $\triangle BAM \sim \triangle FAQ$ and $\triangle CAM \sim \triangle EAQ.$ Thus we have the following relationship:

$$\frac{BM}{FQ} = \frac{MA}{QA} = \frac{CM}{EQ}.$$

Putting these two relationships together, it follows that $BM=CM.$  

We will now prove the other direction. Suppose $BM=MC.$ Then by Ceva's Theorem we have that

\begin{align*} 
\frac{CE}{AE}&=\frac{BF}{AF}\\
\frac{CE}{BF}&=\frac{AE}{AF} = \frac{CE+AE}{BF+AF}=\frac{AC}{AB}\\
\frac{AE}{AC}&=\frac{AF}{AB}.\\
\end{align*}

Since $\angle FAE=\angle BAC,$ we have that $\triangle FAE \sim \triangle BAC.$ Thus $\angle AEF = \angle ACB,$ therefore $\overline{EF} \parallel \overline{BC}.$ \end{proof}



 \textbf{Problem 3.12}.   Give an alternative proof of Lemma 3.9 by taking a negative homothety.

\begin{proof}
  Consider a homothety centered at $G$ with $M=h(A), N=h(B), L = h(C).$ Note that $\triangle ACB \sim \triangle NCM$ by midpoints and that $\triangle {ALG} \sim \triangle {Mh(L)G}$ by homothety. Also notice that $h(L)$ is the midpoint of $NM.$ Since ${AB}/{NM} = 2/1,$

$$\frac{AB}{NM} = \frac{AL}{Mh(L)} = \frac{AG}{MG} = \frac21.$$
\end{proof}



  \textbf{Lemma 3.13 (Euler Line)}.   In triangle $ABC$, prove that $O, G, H$ (with their usual meanings) are collinear and that $G$ divides $\overline{OH}$ in a $2:1$ ratio.

 \begin{proof}  We will first show that $O,G,H$ are collinear. Call the point where the perpendicular from $O$ meets $\overline{BC},\overline{CA}, \overline{AB}$ points $A',B',C',$ respectively. Since $\overline{BC},\overline{CA}, \overline{AB}$ are chords of the circle $(ABC),$ points $A',B',C'$ are in fact the midpoints of their respective line segments. Thus $A'$ lies on $\overline{AG},$ $B'$ lies on $\overline{BG},$ and $C'$ lies on $\overline{CG}.$ Now notice that $\overline{AH} \parallel \overline{OA'}, \overline{BH} \parallel \overline{OB'}, \overline{CH} \parallel \overline{OC'}$ since they are all perpendicular to some side of the triangle $ABC.$  Thus, a homothety $h$ centered at $G$ exists such that $h(A) = A', h(B) = B', h(C) = C'.$ Thus, $h(O) = H,$ so $O,G,H$ are collinear. 

We will now show that $G$ divides $\overline{OH}$ in a $2:1$ ratio. This is equivalent to showing that the homothety $h$ must have a scale factor $k = -2.$ From Lemma 3.9 (Centroid Division) we have that $AG/GA' = 2/1.$ Since $G$ lies in between $A$ and $A',$ we have that $k=-2,$ as desired. (!!!) \end{proof}



  \textbf{Problem 3.16}.   Let $ABC$ be a triangle with contact triangle $DEF$. Prove that $\overline{AD}, \overline{BE}, \overline{CF}$ concur. The point of concurrency is the Gergonne point of triangle $ABC$.

\begin{proof}
  Notice by Lemma 2.15 we have that 

\begin{align*}
AE &= AF = s-a\\
BD &= BF = s-b\\
CD &= CE = s-c.
\end{align*}

Thus, by Ceva's Theorem, we have that

$$\frac{BD}{DC}\cdot\frac{CE}{EA}\cdot\frac{AF}{FB} = \frac{s-b}{s-c}\cdot\frac{s-c}{s-a}\cdot\frac{s-a}{s-b} =1.$$
\end{proof}



  \textbf{Lemma 3.17}.   In cyclic quadrilateral $ABCD$, points $X$ and $Y$ are the orthocenters of $\triangle ABC$ and $\triangle BCD.$ Show that  $AXYD$ is a parallelogram.

 \begin{proof}  Reflect $X$ and $Y$ across $\overline{BC}$ and call these points $X'$ and $Y'$ respectively. Notice that $X'$ and $Y'$ lie on $({ABCD}).$ Thus ${ADX'Y'}$ is a cyclic quadrilateral. Then we have that

$$\measuredangle {AXY} = \measuredangle {X'XY}= \measuredangle {Y'X'X} = \measuredangle {Y'X'A} = \measuredangle {Y'DA} = \measuredangle {YDA}.$$

Similarly, we have that $\measuredangle {DAX} = \measuredangle {XYD}.$ Hence ${AXYD}$ is a parallelogram. \end{proof}



  \textbf{Problem 3.18}.   Let $\overline{AD}, \overline{BE}, \overline{CF}$ be concurrent cevians in a triangle, meeting at $P$. Prove that 

$$\frac{PD}{AD} + \frac{PE}{BE} + \frac{PF}{CF} = 1.$$

 \begin{proof}  By Area Ratios, we can transform each term in our desired equation as follows:

\begin{align*}
  \frac{PD}{AD} &= \frac{[BPC]}{[BAC]}, \\
  \frac{PE}{BE} &= \frac{[CPA]}{[CBA]}, \\
  \frac{PF}{CF} &= \frac{[APB]}{[ACB]}. \\
\end{align*}

Therefore our desired equation turns into

$$\frac{[BPC]}{[BAC]} + \frac{[CPA]}{[CBA]} +\frac{[APB]}{[ACB]} = 1.$$

Notice that $[{BPC}] + [{CPA}] + [{APB}] = [{ABC}].$ Hence we are done. \end{proof} 



 \textbf{Problem 3.19 (Shortlist 2006/G3)}.   Let $ABCDE$ be a convex pentagon such that 

$$\angle BAC = \angle CAD = \angle DAE \text{  and   } \angle ABC = \angle ACD = \angle ADE. $$

Diagonals $BD$ and $CE$ meet at $P$. Prove that ray $AP$ bisects $\overline{CD}$.

 \begin{proof}  Let $B'$ be intersection of diagonals $AC$ and $BD$, and let $E'$ be the intersection of diagonals $AD$ and $CE$. Also let $A'$ be the intersection of ray $AP$ with $CD.$ Notice that the given angle conditions imply that $\triangle ABC \sim \triangle ACD \sim \triangle ADE.$ From this it follows that quadrilaterals $ABCD$ and $ACDE$ are similar. Since $B'$ and $E'$ are the intersections of the diagonals of their respective quadrilaterals, we have that $\frac{CB'}{B'A} = \frac{DE'}{E'A}.$ By Ceva's on $\triangle ACD,$ we have that

$$\frac{AE'}{E'D}\cdot\frac{DA'}{A'C}\cdot\frac{CB'}{B'A} = 1.$$

Since $\frac{CB'}{B'A}\cdot\frac{AE'}{E'D} = 1,$ we have that $DA' = A'C.$ \end{proof}



  \textbf{Problem 3.20 (BAMO 2013/3)}.   Let $H$ be the orthocenter of an acute triangle $ABC$. Consider the circumcenters of triangles $ABH, BCH,$ and $CAH.$ Prove that they are the vertices of a triangle that is congruent to $ABC.$

\begin{proof}
  Let $A',B',C'$ be the circumcenters of $({BCH}),({CAH}),({ABH}),$ respectively. Note that $H$ is the radical center of $({ABH}),({BCH}),({CAH}).$ Thus $\overline{AH} \perp \overline{B'C'}.$ Also notice by properties of circumcenters, $A'$ is on the perpendicular bisector of $\overline{BC}.$ Let $O$ be where the perpendicular bisectors of $\triangle ABC$ intersect (namely, the circumcenter of $\triangle ABC$). Since $\overline{A'O} \parallel \overline{AH},$ $\overline{A'O} \perp \overline{B'C'}.$ This follows similarly for $B'$ and $C',$ hence $O$ is the orthocenter of $\triangle A'B'C'.$ Also notice that, by construction, $H$ is the circumcenter of $\triangle A'B'C'.$ Therefore, a homothety of scale factor $-1$ exists that sends $H$ to $O$, $A$ to $A'$, $B$ to $B'$, and $C$ to $C'$. Hence, $\triangle ABC \cong \triangle A'B'C'.$ 
\end{proof}

\begin{figure}[h]
  \centering
  \begin{asy}
    import graph;
    import olympiad;
    import cse5;
    defaultpen(fontsize(10));
    usepackage("amsmath");
    usepackage("amssymb");
    settings.tex="latex";
    settings.outformat="pdf";
    size(9cm);

    pair A = dir(120);
    pair B = dir(210);
    pair C = dir(330);
    pair H = orthocenter(A,B,C);
    pair O = circumcenter(A,B,C);
    pair Ao = circumcenter(H,B,C);
    pair Bo = circumcenter(H,A,C);
    pair Co = circumcenter(H,A,B);

    draw(A--B--C--cycle, lightred+1.3);
    draw(circumcircle(A,B,C));
    draw(circumcircle(H,B,C));
    draw(circumcircle(H,A,C));
    draw(circumcircle(H,A,B));
    draw(Ao--O);
    draw(Bo--O);
    draw(Co--O);
    draw(Ao--Bo--Co--cycle, lightblue+1.3);

    dot("$A$", A, 2*dir(0));
    dot("$B$", B, 2*dir(-30));
    dot("$C$", C, 2*dir(0));
    dot("$H$", H, dir(-20));
    dot("$O$", O, dir(90));
    dot("$A'$", Ao, dir(Ao));
    dot("$B'$", Bo, dir(Bo));
    dot("$C'$", Co, dir(Co));
    \end{asy}
\caption{Problem 3.20}
\end{figure}

\textbf{Problem 3.21 (USAMO 2003/4)}. Let $ABC$ be a triangle. A circle passing through $A$ and $B$ intersects segments $AC$ and $BC$ at $D$ and $E$, respectively. Lines $AB$ and $DE$ intersect at $F$, while lines $BD$ and $CF$ intersect at $M$. Prove that $MF = MC$ if and only if $MB \cdot MD = MC^2$.

\begin{proof}
  By assuming $MB\cdot  MD = MC^2,$ we have that $\frac{MB}{MC} = \frac{MC}{MD},$ and since $\angle BMC = \angle CMD,$ this implies that $\triangle BMC \sim \triangle CMD$. Since $ABDE$ is a cyclic quadrilateral, $\angle DAE = \angle DBE.$ Now we have that
  $$\angle CAE = \angle DAE = \angle DBE = \angle MBC = \angle MCD = \angle FCA,$$
  hence $\ol{AE} \parallel \ol{CF}.$ Therefore $\triangle ABE \sim \triangle FBC$ and $\frac{FB}{AB} = \frac{CB}{EB}.$ Then 
  \begin{align*}
    \frac{FB}{AB} &= \frac{CB}{EB} \\
    \frac{FA + AB}{AB} &= \frac{CE + EB}{EB} \\
    1 + \frac{FA}{AB} &= 1 + \frac{CE}{EB} \\
    \frac{FA}{AB}&=\frac{CE}{EB}.
  \end{align*}
  By Ceva's on $\triangle BCF$, we have that
  $$\frac{FA}{AB}\cdot\frac{BE}{EC}\cdot\frac{CM}{MF} = 1.$$
  Since $\frac{FA}{AB}=\frac{CE}{EB},$ we have that $MF = MC.$ \\

  We will now go in the reverse direction. We assume $MF = MC.$ By Ceva's on $\triangle BCF,$ 
  $$\frac{FA}{AB}\cdot\frac{BE}{EC}\cdot\frac{CM}{MF} = 1.$$
  and since $MF = MC,$ we have that $\frac{FA}{AB}\cdot\frac{BE}{EC} = 1.$ It follows that
  \begin{align*}
    \frac{FA}{AB} &= \frac{CE}{EB} \\
    1 + \frac{FA}{AB} &= 1 + \frac{CE}{EB}\\
    \frac{AB}{AB} + \frac{FA}{AB} &= \frac{EB}{EB} + \frac{CE}{EB} \\
    \frac{FB}{AB} &= \frac{CB}{EB}. 
  \end{align*}
  Thus $\triangle ABE \sim \triangle FBC.$ This implies that $\ol{AE} \parallel \ol{CF}.$  Since $ABDE$ is a cylic quadrilateral, we have that $\angle FCA = \angle DAE = \angle DBE,$ and since $\angle BMC = \angle CMD,$ we have that $\triangle BMC \sim \triangle CMD$ by $AA\sim.$ Thus $\frac{MB}{MC} = \frac{MC}{MD} \rightarrow MB\cdot MD = MC^2,$ as desired. 
\end{proof}

\textbf{Theorem 3.22 (Monge’s Theorem)}. Consider disjoint circles $\omega_1, \omega_2, \omega_3$ in the plane, no two congruent. For each pair of circles, we construct the intersection of their common external tangents. Prove that these three intersections are collinear. 

\begin{proof}
  Let the points $O_1,O_2,O_3,$ be the centers of $\omega_1, \omega_2, \omega_3$, respectively.
  Let the external tangents of $\omega_1$ and $\omega_2$ meet at $X,$ and define $Y$ and $Z$ analogously. Note that $X,Y,Z$ are each on an extension of a side of $\triangle O_1O_2O_3.$ Let $T_1$ and $T_2$ be points of tangency of $\omega_1$ and $\omega_2$, respectively, where $T_1$ and $T_2$ are on the same side of line $XO_1O_2.$ Note that it is impossible for $X$ to be between $O_1$ and $O_2,$ since $X$ is the intersection of external tangents. Since tangents are always perpendicular to their circles, we have that $\triangle T_1O_1X \sim \triangle T_2O_2X$ by $AA\sim,$ thus with directed lengths we have $\frac{O_1X}{XO_2} = -\frac{r_1}{r_2},$ where $r_1$ and $r_2$ are the radii of $\omega_1$ and $\omega_2.$ Similar arguments can be applied to the other two pairs of circles to give $\frac{O_2Y}{YO_3} = -\frac{r_2}{r_3}$ and $\frac{O_3Z}{ZO_1} = -\frac{r_3}{r_1}.$ Thus
  $$\frac{O_1X}{XO_2}\cdot\frac{O_2Y}{YO_3}\cdot\frac{O_3Z}{ZO_1} = \left(-\frac{r_1}{r_2}\right)\left(-\frac{r_2}{r_3}\right)\left(-\frac{r_3}{r_1}\right) = -1.$$ By Menelaus's Theorem, this proves that $X,Y,Z$ are collinear. 
\end{proof}

\textbf{Theorem 3.23 (Cevian Nest)}. Let $\ol{AX},\ol{BY},\ol{CZ}$ be concurrent cevians of $ABC$. Let $\ol{XD}, \ol{YE},\ol{ZF}$ be concurrent cevians in triangle $XYZ$. Prove that rays $AD,BE,CF$ concur. 

\begin{proof}
  By the Ratio Lemma on $\triangle ZAY,$ we have that
  $$\frac{\sin\angle BAD}{\sin \angle CAD} = \frac{\sin\angle ZAD}{\sin \angle YAD} = \frac{AY}{YC}\cdot\frac{ZD}{DY}.$$
  Similarly, for $\triangle XBZ$ and $\triangle YCX$ we have that
  $$\frac{\sin\angle CBE}{\sin \angle ABE} = \frac{BZ}{XB}\cdot\frac{XE}{EZ}$$ and $$\frac{\sin\angle ACF}{\sin \angle BCF} = \frac{CX}{YC}\cdot\frac{YF}{FX}.$$
  Multiplying these three equations together gives us
  \begin{align*}
  \frac{\sin\angle BAD}{\sin \angle CAD}\cdot\frac{\sin\angle CBE}{\sin \angle ABE}\cdot\frac{\sin\angle ACF}{\sin \angle BCF} &= \left(\frac{AY}{YC}\cdot\frac{CX}{XB}\cdot\frac{BZ}{ZA}\right)\cdot\left(\frac{ZD}{DY}\cdot\frac{YF}{FX}\cdot\frac{XE}{EZ}\right)\\
  &= 1\cdot1 \\
  &= 1.
\end{align*}
  Note that each of the factors in parentheses on the RHS of the first equation are equal to 1 by Ceva's on $\triangle ABC$ and $\triangle XYZ,$ respectively. By the trigonometric form of Ceva's, this implies rays $AD,BE,CF$ concur.
\end{proof}

\textbf{Problem 3.24}. Let $ABC$ be an acute triangle and suppose $X$ is a point on $(ABC)$ with $\ol{AX} \parallel \ol{BC}$ and $X \neq A$. Denote by $G$ the centroid of triangle $ABC$, and by $K$ the foot of the altitude from $A$ to $BC$. Prove that $K,G,X$ are collinear.
\begin{proof}
  Denote by $A',B',C'$ the midpoints of sides $\ol{BC},\ol{CA},\ol{AB}.$ Note that $A',B',C',K$ are on the nine-point circle of $\triangle ABC.$ Also note that each side of $\triangle A'B'C'$ is parallel to a side of $\triangle ABC.$ Therefore there exists a homothety $h$ centered at $G$ such that $h(A) = A', h(B) = B', h(C) = C'.$ Since $\ol{AX} \parallel \ol{BC} \parallel \ol{A'K},$ $h$ sends $K$ to $X.$ Therefore $K,G,X$ are collinear. 
\end{proof}

\begin{figure}[h]
  \centering
  \begin{asy}
    import graph;
    import olympiad;
    import cse5;
    defaultpen(fontsize(10pt));
    usepackage("amsmath");
    usepackage("amssymb");
    settings.tex="latex";
    settings.outformat="pdf";
    size(10cm);

    pair A = dir(120);
    pair B = dir(210);
    pair C = dir(330);
    path ABC = circumcircle(A,B,C);
    draw(A--B--C--cycle);
    draw(ABC);
    pair K = foot(A,B,C);
    pair Aprime = midpoint(B--C);
    pair Bprime = midpoint(C--A);
    pair Cprime = midpoint(A--B);
    draw(circumcircle(Aprime,Bprime,Cprime));
    pair G = centroid(A,B,C);
    draw(A--Aprime);
    draw(B--Bprime);
    draw(C--Cprime);
    draw(Aprime--Bprime);
    draw(Bprime--Cprime);
    draw(Cprime--Aprime);
    path AX = L(A, (1, ypart(A)));
    pair [] X = IPs(AX,ABC);
    pair X = X[1];
    draw(A--X);
    draw(K--X);
    draw(A--K);
    draw(rightanglemark(A,K,C,3));
    draw(rightanglemark(K,A,X,3));

    dot("$A$", A, dir(A));
    dot("$B$", B, dir(B));
    dot("$C$", C, dir(C));
    dot("$K$", K, dir(K));
    dot("$A'$", Aprime, dir(Aprime));
    dot("$B'$", Bprime, dir(Bprime));
    dot("$C'$", Cprime, dir(Cprime));
    dot("$G$", G, dir(0));
    dot("$X$", X, dir(X));
  \end{asy}
  \caption{Problem 3.24}
\end{figure}
\end{document}