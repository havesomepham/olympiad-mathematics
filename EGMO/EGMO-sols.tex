\documentclass[letterpaper,oneside]{scrartcl}

\usepackage[sexy]{evan}
\usepackage{amsmath}
\usepackage{amsthm}
\usepackage{amssymb}
\usepackage{textcomp}
\usepackage{gensymb}

\usepackage{asymptote}

\providecommand{\ol}{\overline}
\providecommand{\ul}{\underline}
\providecommand{\wt}{\widetilde}
\providecommand{\wh}{\widehat}
\providecommand{\eps}{\varepsilon}
\providecommand{\half}{\frac{1}{2}}
\providecommand{\inv}{^{-1}}

\providecommand{\CC}{\mathbb C}
\providecommand{\FF}{\mathbb F}
\providecommand{\NN}{\mathbb N}
\providecommand{\QQ}{\mathbb Q}
\providecommand{\RR}{\mathbb R}
\providecommand{\ZZ}{\mathbb Z}
\providecommand{\ts}{\textsuperscript}
\providecommand{\dg}{^\circ}
\providecommand{\ii}{\item}
\providecommand{\defeq}{\coloneqq}

\providecommand{\hrulebar}{\par
\hspace{\fill}\rule{0.95\linewidth}{.7pt}\hspace{\fill}
\par\nointerlineskip \vspace{\baselineskip}}
\setlength{\parindent}{0pt}

\begin{document}
\title{Euclidean Geometry in Mathematical Olympiads Solutions}
\author{Ryder Pham}
\maketitle

\section{Angle Chasing}
\begin{problem*}
  [1.51, IMO 1985/1]
  A circle has center on the side $\ol{AB}$ of the cyclic quadrilateral $ABCD$. The other three sides are tangent to the circle. Prove that $AD + BC = AB$.
\end{problem*}

\begin{proof}
  \textbf{INCOMPLETE} Call the center of the circle point $O.$ Let the point $T$ be where $AD = AT.$
\end{proof}
\newpage
\section{Circles}

\begin{lemma*}
  [2.19]
  Prove that the A-exradius has length
  $$r_a = \frac{s}{s-a}r,$$
  where $r$ is the inradius.
\end{lemma*}

\begin{proof}
  Drop perpendiculars from $I$ and $I_A$ to $AB$. Call the feet of these perpendiculars $B_1$ and $B_2$ respectively. Notice that $IB_1 = r$ and $I_AB_2 = r_a$ and that $\triangle AB_1I \sim \triangle AB_2I_A$. Therefore

  $$\frac{r}{r_a} = \frac{AB_1}{AB_2},$$

  but by Lemmas 2.15 and 2.17, we know that $AB_1 = s-a$ and $AB_2 = s$, hence

  $$r_a = \frac{s}{s-a}r,$$

  and we are done.
\end{proof}

\begin{lemma*}
  [2.20]
  Let $ABC$ be a triangle. Suppose its incircle and $A$-excircle are tangent to $BC$ at $D$ and $X$, respectively. Show that $BD = CX$ and $BX = CD$.
\end{lemma*}

\begin{proof}  We will first show that $BD=CX$. Let the incircle be tangent to side $AB$ at point $F$ and let to side $AC$ at point $E$. Let the $A$-excircle be tangent to the extension of line $AC$ at $C_1$ and to the extension of line $AB$ at $B_1$. Then

  \begin{align*}
    BD  & = BF                      \\
        & = AB_1 - AF - BB_1        \\
        & = (AC_1 - AE) - BX        \\
        & = (CC_1 + CE) - (BC - CX) \\
        & = CX + (CD - BC) + CX     \\
        & = 2CX - BD                \\
    2BD & =2CX \rightarrow BD = CX.
  \end{align*}

  It follows that $BX=CD$ because

  \begin{align*}
    BD      & = CX         \\
    BD + DX & = DX + CX    \\
    BX      & = CD. \space
  \end{align*}
\end{proof}
\begin{lemma*}
  [2.24]
  Let $ABC$ be a triangle with $I_A, I_B,$ and $I_C$ as excenters. Prove that triangle $I_AI_BI_C$ has orthocenter $I$ and that triangle $ABC$ is its orthic triangle.
\end{lemma*}

\begin{proof}  By the Incenter-Excenter Lemma, we know that $AI_A, BI_B,$ and $CI_C$ coincide at the incenter $I$. We also know from the Lemma that $II_A$ is the diameter of circle $BICI_A.$ Therefore we have that

  $$\angle {I_CCI_A} = \angle {ICI_A} =  90\degree \text{ and } \angle {I_BBI_A} = \angle {IBI_A} = 90\degree .$$

  This follows similarily for $II_B$ and $II_C.$ Now we know that $AI_A, BI_B, CI_C$ are in fact the altitudes of triangle $I_AI_BI_C,$ therefore $I$ is the orthocenter of triangle $I_AI_BI_C.$ Note that since $A,B,$ and $C$ are the feet of the altitudes, $ABC$ is the orthic triangle of triangle $I_AI_BI_C.$ \end{proof}

\begin{theorem*}
  [2.25, The Pitot Theorem]
  Let $ABCD$ be a quadrilateral. If a circle can be inscribed in it, prove that $AB + CD = BC + DA.$
\end{theorem*}

\begin{proof}  Call the points where $AB, BC, CD, DA$ are tangent to the circle $E, F, G, H,$ respectively. Let $AE = AH = a, BE = BF = b, CF = CG = c, DG = DH = d.$ Now note that our condition can be manipulated as follows:

  \begin{align*}
    AB + CD               & = BC + DA               \\
    (AE + BE) + (CG + DG) & = (BF + CF) + (AH + DH) \\
    a + b + c + d         & = b + c + a + d.
  \end{align*}

  Hence, we are done.
\end{proof}

\begin{problem*}
  [2.26, USAMO 1990/5]
  An acute-angled triangle $ABC$ is given in the plane. The circle with diameter $\overline{AB}$ intersects altitude $\overline{CC'}$ and its extension at points $M$ and $N$, and the circle with diameter $\overline{AC}$ intersects altitude $\overline{BB'}$ and its extensions at $P$ and $Q$. Prove that the points $M, N, P , Q$ lie on a common circle.
\end{problem*}

\begin{proof}  Let the circle with diameter $\overline{AB}$ be called $\omega_1$ and the circle with diameter $\overline{AC}$ be called $\omega_2.$  By Theorem 2.9, it suffices to show that the intersection of $\overline{MN}$ and $\overline{PQ}$ lies on the radical axis of $\omega_1$ and $\omega_2.$ Since $\overline{MN}$ and $\overline{PQ}$ are altitudes of $\triangle ABC,$ their intersection is the orthocenter of $\triangle ABC.$ We will call this point $H.$ Note that $\overline{AH}$ is the third altitude of $\triangle ABC.$ Call the foot of this altitude $A'.$ Now note that $\angle AA'B = \angle AA'C = 90\degree,$ and since $\overline{AB}$ and $\overline{AC}$ are diameters of their respective circles, $\omega_1$ and $\omega_2$ must intersect at $A'.$ Hence, $\overline{AA'}$ is the radical axis of $\omega_1$ and $\omega_2,$ and since $A,A',H$ are colinear, $H$ lies on this line. \end{proof}

\begin{figure}[h]
  \centering
  \begin{asy}
    import graph;
    import olympiad;
    import cse5;
    defaultpen(fontsize(10pt));
    usepackage("amsmath");
    usepackage("amssymb");
    settings.tex="latex";
    settings.outformat="pdf";
    size(9cm);

    pair A = dir(120);
    pair B = dir(210);
    pair C = dir(330);
    pair Cprime = foot(C,A,B);
    pair Bprime = foot(B,A,C);
    path AB = A--B;
    path AC = A--C;
    path cAB = Circle(midpoint(AB),distance(midpoint      (AB),A));
    path cAC = Circle(midpoint(AC),distance(midpoint      (AC),A));
    draw(A--B--C--cycle);
    draw(cAB);
    draw(cAC);
    path CCprime =  L(C, Cprime);
    path BBprime =  L(B, Bprime);
    pair []  m = IPs(CCprime, cAB);
    pair []  p = IPs(cAC, BBprime);
    pair M = m[1];
    pair N = m[0];
    pair P = p[0];
    pair Q = p[1];
    draw(M--C);
    draw(P--B);
    draw(circumcircle(M,N,P), dashed);

    pair [] H = IPs(CCprime, BBprime);
    pair H = H[0];
    pair Aprime = foot(A,B,C);
    draw(A--Aprime);
    draw(rightanglemark(B, Cprime, C, 3));
    draw(rightanglemark(B, Bprime, A, 3));
    draw(rightanglemark(A, Aprime, C, 3));

    dot("$A$", A, 1.5dir(90));
    dot("$B$", B, dir(B));
    dot("$C$", C, dir(C));
    dot("$A'$", Aprime, 1.5dir(-100));
    dot("$B'$", Bprime, 2dir(0));
    dot("$C'$", Cprime, dir(30));
    dot("$H$", H, 2dir(20));
    dot("$M$", M, dir(M));
    dot("$N$", N, dir(-90));
    dot("$P$", P, dir(P));
    dot("$Q$", Q, 2dir(-100));
  \end{asy}
  \caption{Problem 2.26}
\end{figure}

\begin{problem*}
  [2.27, BAMO 2012/4]   Given a segment $\overline{AB}$ in the plane, choose on it a point $M$ different from $A$ and $B$. Two equilateral triangles $AMC$ and $BMD$ in the plane are constructed on the same side of segment $\overline{AB}.$ The circumcircles of the two triangles intersect in point $M$ and another point $N.$
  \renewcommand{\labelenumi}{(\alph{enumi})}
  \begin{enumerate}
    \ii Prove that $\overline{AD}$ and $\overline{BC}$ pass through point $N.$

    \ii Prove that no matter where one chooses point $M$ along segment $\overline{AB},$ all lines $MN$ will pass through some fixed point $K$ in the plane.
  \end{enumerate}
\end{problem*}
\begin{proof}  We will prove (a) by angle chasing. Notice that since $ACNM$ and $BDNM$ are cyclic, we have that

  $$\angle AMC = \angle ANC = \angle ACM = \angle ANM = \angle MDB = \angle MNB = 60\degree,$$

  and since $\angle ANC + \angle ANM + \angle MNB = 60\degree + 60\degree + 60\degree = 180\degree,$ we have that $BC$ is a straight line passing through $N.$ A very similar argument follows for $AD.$

  We will now prove (b) using radical axes. First, construct an equilateral triangle $ABE$ on the same side as the other two equilateral triangles. Let the circumcircles around triangles $AMC, BMD,$ and $ABE$ be $\omega_1, \omega_2,$ and $\omega_3,$ respectively. Note that $MN$ is the radical axis of circles $\omega_1$ and $\omega_2,$ the line tangent to circles $\omega_1$ and $\omega_3$ at point $A$ is the radical axis of circles $\omega_1$ and $\omega_3,$ and the line tangent to circles $\omega_2$ and $\omega_3$ at point $B$ is the radical axis of circles $\omega_2$ and $\omega_3.$ Since the centers of  $\omega_1, \omega_2,$ and $\omega_3$ are not colinear, their radical axes (one of which is $MN$) must coincide at the radical center $K.$ Since changing the location of $M$ on $AB$ does not change the tangents at $A$ and $B,$ the point $K$ does not move, hence all possible lines $MN$ must pass through $K.$ \end{proof}



\begin{problem*}
  [2.28, JMO 2012/1]
  Given a triangle $ABC$, let $P$ and $Q$ be points on segments $\overline{AB}$ and $\overline{AC}$, respectively, such that $AP = AQ.$ Let $S$ and $R$ be distinct points on segment $\overline{BC}$ such that $S$ lies between $B$ and $R$, $\angle BPS = \angle PRS$, and $\angle CQR = \angle QSR$. Prove that $P, Q, R, S$ are concyclic.
\end{problem*}

\begin{proof}  Since $\angle BPS = \angle PRS$ by the Tangent Criterion, $\overline{AB}$ is tangent to $(PRS).$ Likewise we have that $\overline{AC}$ is tangent to $(QRS).$ Suppose $(PRS)$ and $(QRS)$ are not the same circle. Then since $AP=AQ$ are both tangents to their respective circles, $A$ must lie on the radical axis $\overline{BC},$ but since $ABC$ is a triangle, this is obviously impossible. Hence $P,Q,R,S$ are concyclic. \end{proof}



\begin{problem*}
  [2.29, IMO 2008/1]
  Let $H$ be the orthocenter of an acute-angled triangle $ABC.$ The circle $\Gamma_A$ centered at the midpoint of $\overline{BC}$ and passing through $H$ intersects the sideline $BC$ at points $A_1$ and $A_2.$ Similarly, define the points $B_1, B_2, C_1,$ and $C_2.$ Prove that six points $A_1, A_2, B_1, B_2, C_1,$ and $C_2$ are concyclic.
\end{problem*}

\begin{proof}
  We will first show that $B_1, B_2, C_1, C_2$ are concyclic. Since $\Gamma_A,\Gamma_B, \Gamma_C$ all intersect at $H$, $H$ is the radical center. We claim that $\overline{AH}$ is the radical axis of $\Gamma_B$ and $\Gamma_C.$ By similar triangles, $M_BM_C$ is parallel to $BC,$ and since $\overline{AH} \perp BC,$ $\overline{AH} \perp M_BM_C.$ The centers of circles $\Gamma_B$ and $\Gamma_C$ are $M_B$ and $M_C,$ respectively, thus $\overline{AH}$ is the radical axis of circles $\Gamma_B$ and $\Gamma_C.$ Since $\overline{B_1B_2}$ and $\overline{C_1C_2}$ intersect at $A,$ by Theorem 2.9 we have shown that $B_1, B_2,C_1,C_2$ are concyclic. Note that the circumcenter of $(B_1B_2C_1C_2)$ is the intersection of the perpendicular bisectors of $B_1B_2$ and $C_1C_2,$ which is the orthocenter $O$ of triangle $ABC.$ Thus what we have proven is that $OB_1 = OB_2 = OC_1 = OC_2.$ A similar argument can be persued for $OA_1$ and $OA_2,$ hence we are done.
\end{proof}

\begin{problem*}
  [2.30, USAMO 1997/2]
  Let $ABC$ be a triangle. Take points $D, E, F$ on the perpendicular bisectors of $\overline{BC}, \overline{CA}, \overline{AB}$ respectively. Show that the lines through $A, B, C$  perpendicular to $\overline{EF}, \overline{FD}, \overline{DE}$ respectively are concurrent.
\end{problem*}

\begin{proof}  Consider the circles with centers $D,E,F$ with chords $BC, CA, AB,$ respectively. Note that the radical axes of these three circles are the lines through $A,B,C$ perpendicular to $\overline{EF}, \overline{FD}, \overline{DE},$ and since the centers of these three circles are not colinear, their radical axes must intersect at a point. \end{proof} (These centers  can  be colinear, but we won't talk about that)

\begin{problem*}
  [2.31, IMO 1995/1]
  Let  $A, B, C, D$ be four distinct points on a line, in that order. The circles with diameters $\overline{AC}$ and $\overline{BD}$ intersect at $X$ and $Y.$ The line $XY$ meets $\overline{BC}$ at $Z.$ Let $P$ be a point on the line $XY$ other than $Z$. The line $CP$ intersects the circle with diameter $AC$ at $C$ and $M$, and the line $BP$ intersects the circle with diameter $BD$ at $B$ and $N$. Prove that the lines $AM, DN, XY$ are concurrent.
\end{problem*}

\begin{proof}  Since $P$ lies on the radical axis of these two circles, and $\overline{BN} \cap \overline{CM} =P,$ $MNBC$ is cyclic by Theorem 2.9. (Reminder that the symbol $\measuredangle$ denotes the directed angle.) Note that

  $$\measuredangle NMC = \measuredangle NBC = \measuredangle NBD = 90\degree - \measuredangle BDN =  90\degree - \measuredangle ADN,$$

  so

  $$\measuredangle NMA =   \measuredangle NMC -90\degree = (90\degree - \measuredangle ADN) - 90\degree = -\measuredangle ADN = \measuredangle NDA,$$

  therefore quadrilateral $DAMN$ is cyclic. The radical axes of the circles $(DAMN), (AMC),$ and $(BND)$ are $\overline{AM}, \overline{DN}, \overline{XY},$ and since the centers of these circles are never colinear, they must intersect at the radical center.
\end{proof}

\begin{problem*}
  [2.32, USAMO 1998/2]
  Let $\mathcal{C}_1$ and $\mathcal{C}_2$ be concentric circles, with $\mathcal{C}_2$ in the interior of $\mathcal{C}_1$. From a point $A$ on $\mathcal{C}_1$ one draws the tangent $\overline{AB}$ to $\mathcal{C}_2$  ($B \in  \mathcal{C}_2$). Let $C$ be the second point of intersection of ray $AB$ and $\mathcal{C}_1$, and let $D$ be the midpoint of $\overline{AB}$. A line passing through $A$ intersects $\mathcal{C}_2$ at $E$ and $F$ in such a way that the perpendicular bisectors of $DE$ and $CF$ intersect at a point $M$ on $AB$. Find, with proof, the ratio $AM/MC$.
\end{problem*}
\begin{proof}  \textbf{INCOMPLETE}
  Note that $CDEF$ is cyclic (need to prove). $M$ is the center of circle $(CDEF).$ Thus $CM=DM.$
\end{proof}
\newpage
\section{Lengths and Ratios}
\begin{theorem*}
  [3.2, Angle Bisector Theorem]   Let $ABC$ be a triangle and $D$ a point on $\overline{BC}$ so that $\overline{AD}$ is the internal angle bisector of $\angle BAC.$ Show that

  $$\frac{AB}{AC} = \frac{DB}{DC}.$$
\end{theorem*}

\begin{proof}  Let $\angle BAD = \alpha = \angle CAD$ and $\angle ADB = \beta.$ Note that $\angle ADC = 180\degree - \beta.$ By Law of Sines, we have

  $$\frac{DB}{\sin\alpha} = \frac{AB}{\sin\beta} \text{  and  } \frac{DC}{\sin\alpha} = \frac{AC}{\sin(180\degree - \beta)}.$$

  Note that $\sin(180\degree-\beta) = \sin\beta.$ Rearranging terms, we have that

  $$\frac{\sin\beta}{\sin\alpha} = \frac{AB}{BD} = \frac{AC}{CD}.$$

  It follows that $\frac{AB}{AC} = \frac{DB}{DC}.$
\end{proof}

\begin{problem*}
  [3.5]
  Show the trigonometric form of Ceva holds.
\end{problem*}

\begin{proof}  Recall that the trigonometric from of Ceva's Theorem is as follows:

  Let $\overline{AX}, \overline{BY}, \overline{CZ}$ be cevians of a triangle $ABC.$ They concur if and only if

  $$\frac{\sin \angle BAX \sin \angle CBY \sin \angle ACZ}{\sin \angle XAC \sin \angle YBA \sin \angle ZCB} = 1.$$

  By the Law of Sines, we have that

  $$\frac{\sin\angle BAX}{BX} = \frac{\sin B}{AX}$$

  and

  $$\frac{\sin\angle XAC}{XC} = \frac{\sin C}{AX}.$$

  Combining these two equations gives us

  $$AX = \frac{BX \sin B}{\sin\angle BAX}=\frac{XC \sin C}{\sin\angle XAC} \Rightarrow \frac{\sin \angle BAX}{\sin \angle XAC} = \frac{BX}{XC} \cdot \frac{\sin C}{\sin B}.$$

  Similarly, we have that

  $$\frac{\sin \angle CBY}{\sin \angle YBA} = \frac{CY}{YA}\cdot \frac{\sin A}{\sin C} $$

  and

  $$
    \frac{\sin \angle ACZ}{\sin \angle ZCB} = \frac{AZ}{ZB}\cdot \frac{\sin B}{\sin A}. $$

  Plugging these values into the original equation, we have that

  $$\frac{BX}{XC}\cdot \frac{CY}{YA} \cdot \frac{AZ}{ZB} = 1,$$

  and we know this is true from the original statement of Ceva's Theorem.
\end{proof}

\begin{problem*}
  [3.6]
  Let $\overline{AM}, \overline{BE},$ and $\overline{CF}$ be concurrent cevians of a triangle $ABC.$ Show that $\overline{EF} \parallel \overline{BC}$ if and only if $BM = MC.$
\end{problem*}

\begin{proof}
  Suppose $\overline{EF} \parallel \overline{BC}.$ Call the point where $\overline{AM}$ intersects $\overline{EF}$ point $Q.$ Notice that $\triangle BPM \sim \triangle EPQ$ and $\triangle CPM \sim \triangle FPQ.$ Thus we have the following relationship:

  $$\frac{BM}{EQ} = \frac{MP}{QP} = \frac{CM}{FQ}.$$

  Now also notice that $\triangle BAM \sim \triangle FAQ$ and $\triangle CAM \sim \triangle EAQ.$ Thus we have the following relationship:

  $$\frac{BM}{FQ} = \frac{MA}{QA} = \frac{CM}{EQ}.$$

  Putting these two relationships together, it follows that $BM=CM.$

  We will now prove the other direction. Suppose $BM=MC.$ Then by Ceva's Theorem we have that

  \begin{align*}
    \frac{CE}{AE} & =\frac{BF}{AF}                                     \\
    \frac{CE}{BF} & =\frac{AE}{AF} = \frac{CE+AE}{BF+AF}=\frac{AC}{AB} \\
    \frac{AE}{AC} & =\frac{AF}{AB}.                                    \\
  \end{align*}

  Since $\angle FAE=\angle BAC,$ we have that $\triangle FAE \sim \triangle BAC.$ Thus $\angle AEF = \angle ACB,$ therefore $\overline{EF} \parallel \overline{BC}.$
\end{proof}

\begin{problem*}
  [3.12]
  Give an alternative proof of Lemma 3.9 by taking a negative homothety.
\end{problem*}
\begin{proof}
  Consider a homothety centered at $G$ with $M=h(A), N=h(B), L = h(C).$ Note that $\triangle ACB \sim \triangle NCM$ by midpoints and that $\triangle {ALG} \sim \triangle {Mh(L)G}$ by homothety. Also notice that $h(L)$ is the midpoint of $NM.$ Since ${AB}/{NM} = 2/1,$

  $$\frac{AB}{NM} = \frac{AL}{Mh(L)} = \frac{AG}{MG} = \frac21.$$
\end{proof}



\begin{lemma*}
  [3.13, Euler Line]
  In triangle $ABC$, prove that $O, G, H$ (with their usual meanings) are collinear and that $G$ divides $\overline{OH}$ in a $2:1$ ratio.
\end{lemma*}

\begin{proof}  We will first show that $O,G,H$ are collinear. Call the point where the perpendicular from $O$ meets $\overline{BC},\overline{CA}, \overline{AB}$ points $A',B',C',$ respectively. Since $\overline{BC},\overline{CA}, \overline{AB}$ are chords of the circle $(ABC),$ points $A',B',C'$ are in fact the midpoints of their respective line segments. Thus $A'$ lies on $\overline{AG},$ $B'$ lies on $\overline{BG},$ and $C'$ lies on $\overline{CG}.$ Now notice that $\overline{AH} \parallel \overline{OA'}, \overline{BH} \parallel \overline{OB'}, \overline{CH} \parallel \overline{OC'}$ since they are all perpendicular to some side of the triangle $ABC.$  Thus, a homothety $h$ centered at $G$ exists such that $h(A) = A', h(B) = B', h(C) = C'.$ Thus, $h(O) = H,$ so $O,G,H$ are collinear.

  We will now show that $G$ divides $\overline{OH}$ in a $2:1$ ratio. This is equivalent to showing that the homothety $h$ must have a scale factor $k = -2.$ From Lemma 3.9 (Centroid Division) we have that $AG/GA' = 2/1.$ Since $G$ lies in between $A$ and $A',$ we have that $k=-2,$ as desired. (!!!) \end{proof}



\begin{problem*}
  [3.16]
  Let $ABC$ be a triangle with contact triangle $DEF$. Prove that $\overline{AD}, \overline{BE}, \overline{CF}$ concur. The point of concurrency is the Gergonne point of triangle $ABC$.
\end{problem*}

\begin{proof}
  Notice by Lemma 2.15 we have that

  \begin{align*}
    AE & = AF = s-a  \\
    BD & = BF = s-b  \\
    CD & = CE = s-c.
  \end{align*}

  Thus, by Ceva's Theorem, we have that

  $$\frac{BD}{DC}\cdot\frac{CE}{EA}\cdot\frac{AF}{FB} = \frac{s-b}{s-c}\cdot\frac{s-c}{s-a}\cdot\frac{s-a}{s-b} =1.$$
\end{proof}



\begin{lemma*}
  [3.17]
  In cyclic quadrilateral $ABCD$, points $X$ and $Y$ are the orthocenters of $\triangle ABC$ and $\triangle BCD.$ Show that  $AXYD$ is a parallelogram.
\end{lemma*}

\begin{proof}  Reflect $X$ and $Y$ across $\overline{BC}$ and call these points $X'$ and $Y'$ respectively. Notice that $X'$ and $Y'$ lie on $({ABCD}).$ Thus ${ADX'Y'}$ is a cyclic quadrilateral. Then we have that

  $$\measuredangle {AXY} = \measuredangle {X'XY}= \measuredangle {Y'X'X} = \measuredangle {Y'X'A} = \measuredangle {Y'DA} = \measuredangle {YDA}.$$

  Similarly, we have that $\measuredangle {DAX} = \measuredangle {XYD}.$ Hence ${AXYD}$ is a parallelogram.
\end{proof}

\begin{problem*}
  [3.18]
  Let $\overline{AD}, \overline{BE}, \overline{CF}$ be concurrent cevians in a triangle, meeting at $P$. Prove that

  $$\frac{PD}{AD} + \frac{PE}{BE} + \frac{PF}{CF} = 1.$$
\end{problem*}

\begin{proof}  By Area Ratios, we can transform each term in our desired equation as follows:

  \begin{align*}
    \frac{PD}{AD} & = \frac{[BPC]}{[BAC]}, \\
    \frac{PE}{BE} & = \frac{[CPA]}{[CBA]}, \\
    \frac{PF}{CF} & = \frac{[APB]}{[ACB]}. \\
  \end{align*}

  Therefore our desired equation turns into

  $$\frac{[BPC]}{[BAC]} + \frac{[CPA]}{[CBA]} +\frac{[APB]}{[ACB]} = 1.$$

  Notice that $[{BPC}] + [{CPA}] + [{APB}] = [{ABC}].$ Hence we are done. \end{proof}

\begin{problem*}
  [3.19, Shortlist 2006/G3]
  Let $ABCDE$ be a convex pentagon such that

  $$\angle BAC = \angle CAD = \angle DAE \text{  and   } \angle ABC = \angle ACD = \angle ADE. $$

  Diagonals $BD$ and $CE$ meet at $P$. Prove that ray $AP$ bisects $\overline{CD}$.

\end{problem*}
\begin{proof}  Let $B'$ be intersection of diagonals $AC$ and $BD$, and let $E'$ be the intersection of diagonals $AD$ and $CE$. Also let $A'$ be the intersection of ray $AP$ with $CD.$ Notice that the given angle conditions imply that $\triangle ABC \sim \triangle ACD \sim \triangle ADE.$ From this it follows that quadrilaterals $ABCD$ and $ACDE$ are similar. Since $B'$ and $E'$ are the intersections of the diagonals of their respective quadrilaterals, we have that $\frac{CB'}{B'A} = \frac{DE'}{E'A}.$ By Ceva's on $\triangle ACD,$ we have that

  $$\frac{AE'}{E'D}\cdot\frac{DA'}{A'C}\cdot\frac{CB'}{B'A} = 1.$$

  Since $\frac{CB'}{B'A}\cdot\frac{AE'}{E'D} = 1,$ we have that $DA' = A'C.$ \end{proof}



\begin{problem*}
  [3.20 (BAMO 2013/3)]
  Let $H$ be the orthocenter of an acute triangle $ABC$. Consider the circumcenters of triangles $ABH, BCH,$ and $CAH.$ Prove that they are the vertices of a triangle that is congruent to $ABC.$
\end{problem*}

\begin{proof}
  Let $A',B',C'$ be the circumcenters of $({BCH}),({CAH}),({ABH}),$ respectively. Note that $H$ is the radical center of $({ABH}),({BCH}),({CAH}).$ Thus $\overline{AH} \perp \overline{B'C'}.$ Also notice by properties of circumcenters, $A'$ is on the perpendicular bisector of $\overline{BC}.$ Let $O$ be where the perpendicular bisectors of $\triangle ABC$ intersect (namely, the circumcenter of $\triangle ABC$). Since $\overline{A'O} \parallel \overline{AH},$ $\overline{A'O} \perp \overline{B'C'}.$ This follows similarly for $B'$ and $C',$ hence $O$ is the orthocenter of $\triangle A'B'C'.$ Also notice that, by construction, $H$ is the circumcenter of $\triangle A'B'C'.$ Therefore, a homothety of scale factor $-1$ exists that sends $H$ to $O$, $A$ to $A'$, $B$ to $B'$, and $C$ to $C'$. Hence, $\triangle ABC \cong \triangle A'B'C'.$
\end{proof}

\begin{figure}[h]
  \centering
  \begin{asy}
    import graph;
    import olympiad;
    import cse5;
    defaultpen(fontsize(10));
    usepackage("amsmath");
    usepackage("amssymb");
    settings.tex="latex";
    settings.outformat="pdf";
    size(9cm);

    pair A = dir(120);
    pair B = dir(210);
    pair C = dir(330);
    pair H = orthocenter(A,B,C);
    pair O = circumcenter(A,B,C);
    pair Ao = circumcenter(H,B,C);
    pair Bo = circumcenter(H,A,C);
    pair Co = circumcenter(H,A,B);

    draw(A--B--C--cycle, lightred+1.3);
    draw(circumcircle(A,B,C));
    draw(circumcircle(H,B,C));
    draw(circumcircle(H,A,C));
    draw(circumcircle(H,A,B));
    draw(Ao--O);
    draw(Bo--O);
    draw(Co--O);
    draw(Ao--Bo--Co--cycle, lightblue+1.3);

    dot("$A$", A, 2*dir(0));
    dot("$B$", B, 2*dir(-30));
    dot("$C$", C, 2*dir(0));
    dot("$H$", H, dir(-20));
    dot("$O$", O, dir(90));
    dot("$A'$", Ao, dir(Ao));
    dot("$B'$", Bo, dir(Bo));
    dot("$C'$", Co, dir(Co));
  \end{asy}
  \caption{Problem 3.20}
\end{figure}

\begin{problem*}
  [3.21 (USAMO 2003/4)]
  Let $ABC$ be a triangle. A circle passing through $A$ and $B$ intersects segments $AC$ and $BC$ at $D$ and $E$, respectively. Lines $AB$ and $DE$ intersect at $F$, while lines $BD$ and $CF$ intersect at $M$. Prove that $MF = MC$ if and only if $MB \cdot MD = MC^2$.
\end{problem*}

\begin{proof}
  By assuming $MB\cdot  MD = MC^2,$ we have that $\frac{MB}{MC} = \frac{MC}{MD},$ and since $\angle BMC = \angle CMD,$ this implies that $\triangle BMC \sim \triangle CMD$. Since $ABDE$ is a cyclic quadrilateral, $\angle DAE = \angle DBE.$ Now we have that
  $$\angle CAE = \angle DAE = \angle DBE = \angle MBC = \angle MCD = \angle FCA,$$
  hence $\ol{AE} \parallel \ol{CF}.$ Therefore $\triangle ABE \sim \triangle FBC$ and $\frac{FB}{AB} = \frac{CB}{EB}.$ Then
  \begin{align*}
    \frac{FB}{AB}      & = \frac{CB}{EB}      \\
    \frac{FA + AB}{AB} & = \frac{CE + EB}{EB} \\
    1 + \frac{FA}{AB}  & = 1 + \frac{CE}{EB}  \\
    \frac{FA}{AB}      & =\frac{CE}{EB}.
  \end{align*}
  By Ceva's on $\triangle BCF$, we have that
  $$\frac{FA}{AB}\cdot\frac{BE}{EC}\cdot\frac{CM}{MF} = 1.$$
  Since $\frac{FA}{AB}=\frac{CE}{EB},$ we have that $MF = MC.$ \\

  We will now go in the reverse direction. We assume $MF = MC.$ By Ceva's on $\triangle BCF,$
  $$\frac{FA}{AB}\cdot\frac{BE}{EC}\cdot\frac{CM}{MF} = 1.$$
  and since $MF = MC,$ we have that $\frac{FA}{AB}\cdot\frac{BE}{EC} = 1.$ It follows that
  \begin{align*}
    \frac{FA}{AB}                 & = \frac{CE}{EB}                 \\
    1 + \frac{FA}{AB}             & = 1 + \frac{CE}{EB}             \\
    \frac{AB}{AB} + \frac{FA}{AB} & = \frac{EB}{EB} + \frac{CE}{EB} \\
    \frac{FB}{AB}                 & = \frac{CB}{EB}.
  \end{align*}
  Thus $\triangle ABE \sim \triangle FBC.$ This implies that $\ol{AE} \parallel \ol{CF}.$  Since $ABDE$ is a cylic quadrilateral, we have that $\angle FCA = \angle DAE = \angle DBE,$ and since $\angle BMC = \angle CMD,$ we have that $\triangle BMC \sim \triangle CMD$ by $AA\sim.$ Thus $\frac{MB}{MC} = \frac{MC}{MD} \rightarrow MB\cdot MD = MC^2,$ as desired.
\end{proof}

\begin{theorem*}
  [3.22, Monge’s Theorem]
  Consider disjoint circles $\omega_1, \omega_2, \omega_3$ in the plane, no two congruent. For each pair of circles, we construct the intersection of their common external tangents. Prove that these three intersections are collinear.
\end{theorem*}

\begin{proof}
  Let the points $O_1,O_2,O_3,$ be the centers of $\omega_1, \omega_2, \omega_3$, respectively.
  Let the external tangents of $\omega_1$ and $\omega_2$ meet at $X,$ and define $Y$ and $Z$ analogously. Note that $X,Y,Z$ are each on an extension of a side of $\triangle O_1O_2O_3.$ Let $T_1$ and $T_2$ be points of tangency of $\omega_1$ and $\omega_2$, respectively, where $T_1$ and $T_2$ are on the same side of line $XO_1O_2.$ Note that it is impossible for $X$ to be between $O_1$ and $O_2,$ since $X$ is the intersection of external tangents. Since tangents are always perpendicular to their circles, we have that $\triangle T_1O_1X \sim \triangle T_2O_2X$ by $AA\sim,$ thus with directed lengths we have $\frac{O_1X}{XO_2} = -\frac{r_1}{r_2},$ where $r_1$ and $r_2$ are the radii of $\omega_1$ and $\omega_2.$ Similar arguments can be applied to the other two pairs of circles to give $\frac{O_2Y}{YO_3} = -\frac{r_2}{r_3}$ and $\frac{O_3Z}{ZO_1} = -\frac{r_3}{r_1}.$ Thus
  $$\frac{O_1X}{XO_2}\cdot\frac{O_2Y}{YO_3}\cdot\frac{O_3Z}{ZO_1} = \left(-\frac{r_1}{r_2}\right)\left(-\frac{r_2}{r_3}\right)\left(-\frac{r_3}{r_1}\right) = -1.$$ By Menelaus's Theorem, this proves that $X,Y,Z$ are collinear.
\end{proof}

\begin{theorem*}
  [3.23, Cevian Nest]
  Let $\ol{AX},\ol{BY},\ol{CZ}$ be concurrent cevians of $ABC$. Let $\ol{XD}, \ol{YE},\ol{ZF}$ be concurrent cevians in triangle $XYZ$. Prove that rays $AD,BE,CF$ concur.
\end{theorem*}

\begin{proof}
  By the Ratio Lemma on $\triangle ZAY,$ we have that
  $$\frac{\sin\angle BAD}{\sin \angle CAD} = \frac{\sin\angle ZAD}{\sin \angle YAD} = \frac{AY}{YC}\cdot\frac{ZD}{DY}.$$
  Similarly, for $\triangle XBZ$ and $\triangle YCX$ we have that
  $$\frac{\sin\angle CBE}{\sin \angle ABE} = \frac{BZ}{XB}\cdot\frac{XE}{EZ}$$ and $$\frac{\sin\angle ACF}{\sin \angle BCF} = \frac{CX}{YC}\cdot\frac{YF}{FX}.$$
  Multiplying these three equations together gives us
  \begin{align*}
    \frac{\sin\angle BAD}{\sin \angle CAD}\cdot\frac{\sin\angle CBE}{\sin \angle ABE}\cdot\frac{\sin\angle ACF}{\sin \angle BCF} & = \left(\frac{AY}{YC}\cdot\frac{CX}{XB}\cdot\frac{BZ}{ZA}\right)\cdot\left(\frac{ZD}{DY}\cdot\frac{YF}{FX}\cdot\frac{XE}{EZ}\right) \\
                                                                                                                                 & = 1\cdot1                                                                                                                           \\
                                                                                                                                 & = 1.
  \end{align*}
  Note that each of the factors in parentheses on the RHS of the first equation are equal to 1 by Ceva's on $\triangle ABC$ and $\triangle XYZ,$ respectively. By the trigonometric form of Ceva's, this implies rays $AD,BE,CF$ concur.
\end{proof}

\begin{problem*}
  [3.24]
  Let $ABC$ be an acute triangle and suppose $X$ is a point on $(ABC)$ with $\ol{AX} \parallel \ol{BC}$ and $X \neq A$. Denote by $G$ the centroid of triangle $ABC$, and by $K$ the foot of the altitude from $A$ to $BC$. Prove that $K,G,X$ are collinear.
\end{problem*}
\begin{proof}
  Denote by $A',B',C'$ the midpoints of sides $\ol{BC},\ol{CA},\ol{AB}.$ Note that $A',B',C',K$ are on the nine-point circle of $\triangle ABC.$ Also note that each side of $\triangle A'B'C'$ is parallel to a side of $\triangle ABC.$ Therefore there exists a homothety $h$ centered at $G$ such that $h(A) = A', h(B) = B', h(C) = C'.$ Since $\ol{AX} \parallel \ol{BC} \parallel \ol{A'K},$ $h$ sends $K$ to $X.$ Therefore $K,G,X$ are collinear. \\
\end{proof}

\begin{figure}[t]
  \centering
  \begin{asy}
    import graph;
    import olympiad;
    import cse5;
    defaultpen(fontsize(10pt));
    usepackage("amsmath");
    usepackage("amssymb");
    settings.tex="latex";
    settings.outformat="pdf";
    size(7cm);

    pair A = dir(120);
    pair B = dir(210);
    pair C = dir(330);
    path ABC = circumcircle(A,B,C);
    draw(A--B--C--cycle);
    draw(ABC);
    pair K = foot(A,B,C);
    pair Aprime = midpoint(B--C);
    pair Bprime = midpoint(C--A);
    pair Cprime = midpoint(A--B);
    draw(circumcircle(Aprime,Bprime,Cprime));
    pair G = centroid(A,B,C);
    draw(A--Aprime);
    draw(B--Bprime);
    draw(C--Cprime);
    draw(Aprime--Bprime);
    draw(Bprime--Cprime);
    draw(Cprime--Aprime);
    path AX = L(A, (1, ypart(A)));
    pair [] X = IPs(AX,ABC);
    pair X = X[1];
    draw(A--X);
    draw(K--X);
    draw(A--K);
    draw(rightanglemark(A,K,C,3));
    draw(rightanglemark(K,A,X,3));

    dot("$A$", A, dir(A));
    dot("$B$", B, dir(B));
    dot("$C$", C, dir(C));
    dot("$K$", K, dir(K));
    dot("$A'$", Aprime, dir(Aprime));
    dot("$B'$", Bprime, dir(Bprime));
    dot("$C'$", Cprime, dir(Cprime));
    dot("$G$", G, dir(0));
    dot("$X$", X, dir(X));
  \end{asy}
  \caption{Problem 3.24}
\end{figure}

\begin{problem*}
  [3.25 (USAMO 1993/2)]
  Let $ABCD$ be a quadrilateral whose diagonals $\ol{AC}$ and $\ol{BD}$ are perpendicular and intersect at $E$. Prove that the reflections of $E$ across $\ol{AB},\ol{BC},\ol{CD},\ol{DA}$ are concyclic.
\end{problem*}

\begin{proof}
  Denote by $P,Q,R,S$ the projections of $E$ onto $\ol{AB},\ol{BC},\ol{CD},\ol{DA},$ respectively. Note that quadrilaterals $APES, BQEP, CREQ, DSER$ are cyclic. We will show $PQRS$ is cyclic by angle chase.
  \begin{align*}
    \dang SPQ & = \dang SPE + \dang EPQ                         \\
              & = \dang SAE + \dang EBQ                         \\
              & = 90\degree - \dang EDS + 90\degree - \dang QCE \\
              & = \dang SRE + \dang ERQ                         \\
              & = \dang SRQ.
  \end{align*}
  A homothety centered at $E$ with a scale factor of 2 sends $PQRS$ to the desired quadrialteral. Hence we are done.
\end{proof}

\begin{problem*}
  [3.26 (EGMO 2013/1)]
  The side $BC$ of the triangle $ABC$ is extended beyond $C$ to $D$ so that $CD=BC$. The side $CA$ is extended beyond $A$ to $E$ so that $AE=2CA$. Prove that if $AD=BE$ then the triangle $ABC$ is right-angled.
\end{problem*}

\begin{proof}
  Let $BC=a, CA=b, AB=c,\angle BAC = \theta.$ Thus $CD = a$ and $AE = 2b.$ Let $F$ be a point on $\ol{AB}$ such that $\ol{DF} \parallel \ol{AC}.$ By similar triangles, $AF=c, DF = 2b,$ and $\angle AFD = \theta.$ We also have that $\angle EAC = 180\degree - \theta.$ Thus by the Law of Cosines,
  $${BE}^2 = (2b)^2+c^2-2\cdot 2b \cdot c\cdot\cos(180\degree - \theta)$$ and $${AD}^2 = (2b)^2+c^2-2\cdot 2b \cdot c\cdot\cos(\theta).$$
  Since $AD = BE,$ we have that $\cos(180\degree - \theta) = \cos(\theta).$ Therefore $\theta = 90\degree,$ as desired.
\end{proof}

\begin{problem*}
  [3.27 (APMO 2004/2)]
  Let $O$ be the circumcenter and $H$ the orthocenter of an acute triangle $ABC$. Prove that the area of one of the triangles $AOH, BOH,$ and $COH$ is equal to the sum of the areas of the other two.
\end{problem*}
\begin{proof}
  (NOT ORIGINAL) WLOG let $B$ and $C$ be on the same side of line $\ol{OH}.$ Let $M$ be the midpoint of $BC.$ Denote by $A',B',C',M'$ the projections onto $OH$ of $A,B,C,M,$ respectively. Notice that a homothety centered at $G$ with a ratio of $-2$ sends $\triangle MM'G$ to $\triangle AA'G,$ so $AA' = 2MM' = BB' + CC',$ which implies the result.
\end{proof}

\begin{problem*}
  [3.28 (Shortlist 2001/G1)]
  Let $A_1$ be the center of the square inscribed in acute triangle $ABC$ with two vertices of the square on side $BC$. Thus one of the two remaining vertices of the square is on side $AB$ and the other is on $AC$. Points $B_1$ and $C_1$ are defined in a similar way for inscribed squares with two vertices on sides $AC$ and $AB$, respectively. Prove that lines $AA_1, BB_1, CC_1$ are concurrent.
\end{problem*}

\begin{proof}
  (NOT ENTIRELY ORIGINAL) Denote by $A_{\square}$ the square with center $A_1$. Denote by $P,Q$ the vertices of $A_{\square}$ on sides $AB$ and $AC,$ respectively. Consider a homothety centered at $A$ that sends $P$ to $B$ and $Q$ to $C.$ The center $A'_1,$ of the new square lies on $AA_1.$ Similar arguments hold for homotheties centered at $B$ and $C.$ Thus we just need to show that $AA'_1, BB'_1, CC'_1$ are concurrent. By the Law of Sines,
  $$\frac{A'_1B}{\sin\angle A'_1AB} = \frac{AA'_1}{\sin(B+45\degree)} \text{ and } \frac{A'_1C}{\sin\angle A'_1AC} = \frac{AA'_1}{\sin(C+45\degree)}.$$
  Since $A'_1B = A'_1C,$ we have that $\displaystyle \frac{\sin(B+45\degree)}{\sin(C+45\degree)} = \frac{\sin\angle A'_1AB}{\sin\angle A'_1AC}.$ Doing this for $B'_1$ and $C'_1$ then multiplying all three equations together proves the desired conclusion by the Trigonometric form of Ceva's Theorem.
\end{proof}
\newpage
\section{Assorted Configurations}
\begin{proposition*}
  Proposition [4.1]
  Prove that the Simson line is parallel to $\ol{AK}$ in the notation of Figure 4.1A.
\end{proposition*}
\begin{proof}
  Notice that $BXPZ$ is cyclic since $\dang BZP = \dang BXP$. Then
  $$\dang YXP = \dang ZXP = \dang ZBP  = \dang ABP = \dang AKP.$$
\end{proof}
\begin{problem*}
  [4.2]
  Let $K'$ be the reflection of $K$ across $\ol{BC}$. Show that $K'$ is the orthocenter of $\triangle PBC$.
\end{problem*}
\begin{proof}
  Note that $K'$ is already on the altitude $PX.$ Also note that $K$ is on the circumcircle of $\triangle PBC.$ Since $K$ and $K'$ are reflections of each other over $BC,$ by Lemma 1.17 we have that $K'$ is the orthocenter of $\triangle PBC.$
\end{proof}
\begin{problem*}
  [4.3]
  Show that $LHXP$ is a parallelogram.
\end{problem*}
\begin{proof}
  Note that $\ol{LH} \parallel \ol{XP}$ by construction. Thus it suffices to show that $LH = XP.$ By Proposition 4.1 we have that $LA = XK.$ Also, by the conclusion made after Problem 4.2 we have that $AH = PK'.$ Thus
  $$LH = LA + AH = XK + PK' = XK' + PK' = XP.$$
\end{proof}

\begin{problem*}
  [4.5]
  Check $\angle IAI_B = 90\degree$ and $\angle IAI_C = 90\degree.$
\end{problem*}
\begin{proof}
  By the Incenter-Excenter Lemma we know that $II_B$ is the diameter of circle $(AICI_B)$. Therefore $\angle IAI_B = 90\degree.$ A similar argument holds to show $\angle IAI_C = 90\degree.$
\end{proof}

\begin{problem*}
  [4.7]
  How are Lemma 1.18, Lemma 3.11, and Lemma 4.6 related?
\end{problem*}
Let $L$ be the midpoint of $II_A.$ Then in Figure 4.2A, $(ABCL)$ is in fact the nine-point circle of $\triangle I_AI_BI_C$ (Lemma 3.11). Moreover, by Lemma 1.18, $I,B,C,I_A$ all lie on a circle centered at $L.$ Finally, by Lemma 4.6, we know $I$ is the orthocenter of $\triangle I_AI_BI_C,$ but we also can derive this from Lemma 3.11. This means that any two lemmas can prove the third.

\begin{problem*}
  [4.8]
  Prove that $A, E$, and $X$ are collinear and that $\ol{DE}$ is a diameter of the incircle.
\end{problem*}
\begin{proof}
  Consider the homothety bringing $\triangle AB'C'$ to $\triangle ABC.$ This homothety brings the circle with center $I$ to the circle with center $I_A,$ so it must bring point $E$ to point $X.$ Hence $A,E,X$ are collinear. Now notice that $\ol{BC} \perp \ol{ID}$ and $\ol{B'C'} \perp \ol{IE}$ by tangency and $\ol{B'C'} \parallel \ol{BC}.$ This is enough to show $I,D,E$ are collinear, therefore $\ol{DE}$ is a diameter of the incircle.
\end{proof}

\begin{lemma*}
  [4.10, Diameter of the Excircle]
  In the notation of Lemma 4.9, suppose $\ol{XY}$ is a diameter of the $A$-excircle. Show that $D$ lies on $\ol{AY}.$
\end{lemma*}
\begin{proof}
  Consider again the homothety bringing $\triangle AB'C'$ to $\triangle ABC.$ Since $DE$ and $YX$ are diameters of the incircle and $A$-excircle, respectively, the homothety brings $D$ to $Y.$ Hence $D$ lies on $\ol{AY}.$
\end{proof}

\begin{problem*}
  [4.11]
  If $M$ is the midpoint of $\ol{BC},$ prove that $\ol{AE} \parallel \ol{IM}.$
\end{problem*}
\begin{proof}
  Since $M$ is the midpoint of $BC,$ we know that $M$ is also the midpoint of $DX$ since $BD=XC.$ We also know that $I$ is the midpoint of $DE.$ Thus we find that $\triangle DEX \sim \triangle DIM,$ implying the result.
\end{proof}

\begin{problem*}
  [4.12]
  Prove that points $X,I,M$ are collinear.
\end{problem*}
\begin{proof}
  Note that $AK \parallel ED,$ so the homothety centered at $X$ that sends $\triangle AXK$ to $\triangle EXD$ sends $M$ to $I$ because they are midpoints of $AK$ and $ED,$ respectively. This implies that $X,I,M$ are collinear.
\end{proof}

\begin{problem*}
  [4.13]
  Show that $D,I_A,M$ are collinear.
\end{problem*}
\begin{proof}
  Note that $AK \parallel XY,$ so the homothety centered at $D$ that sends $\triangle XDY$ to $\triangle KDA$ sends $I_A$ to $M$ since they are midpoints of $XY$ and $AK,$ respectively. This implies that $D,I_A,M$ are collinear.
\end{proof}

\begin{problem*}
  [4.15]
  Show that $I$ must lie on $(AB'C').$
\end{problem*}
\begin{proof}
  Since $B'C' \parallel BC, \angle IXB' = \angle IXC' = 90\degree.$ Also note that $\angle BFI = 90\degree = \angle IEC$ by tangency. Thus $\ol{FXE}$ is the Simson line of $I$ with respect to $\triangle AB'C'.$ Thus by Lemma 1.48 we are done.
\end{proof}

\begin{problem*}
  [4.16]
  Prove that $XB'=XC'.$
\end{problem*}
\begin{proof}
  Note that since $AI$ bisects $\angle B'AC',$ and by Problem 4.15 $I$ lies on $(AB'C'),$ then $I$ must be the midpoint of arc $B'IC'$ by the Incenter-Excenter Lemma. Since $X$ is the foot of the altitude from $I$ to $B'C',$ and $IB'=IC',$ it follows that $XB' = XC'.$
\end{proof}

\begin{problem*}
  [4.19]
  Show that if two of the angle relations in Lemma 4.18 hold, then so does the third.
\end{problem*}
\begin{proof}
  Note by the trigonometric form of Ceva's Theorem we have that
  $$\frac{\sin\dang BAP \cdot \sin\dang CBP \cdot \sin\dang ACP}{\sin\dang PAC \cdot \sin\dang PBA \cdot \sin\dang PCB} = 1$$ and
  $$\frac{\sin\dang P^*AC \cdot \sin\dang P^*BA \cdot \sin\dang P^*CB}{\sin\dang BAP^* \cdot \sin\dang CBP^* \cdot \sin\dang ACP^*} = 1.$$
  Now WLOG assume $\dang BAP = \dang P^*AC$ and $\dang CBP = \dang P^*BA.$ This implies that both $\dang PAC = \dang BAP^*$ and $\dang PBA = \dang CBP^*.$ Thus equating the two above equations and simplifying gives us
  $$\frac{\sin\dang ACP}{\sin\dang PCB} = \frac{\sin \dang P^*CB}{\sin\dang ACP^*}.$$ Noticing that $\dang ACP + \dang PCB = \dang ACB = \dang ACP^*+\dang P^*CB,$ we can easily see that this implies $\dang ACP = \dang P^*CB.$
\end{proof}

\begin{problem*}
  [4.20]
  Prove that the cevians $AX', BY',$ and $CZ'$ concur as described above.
\end{problem*}
\begin{proof}
  By Ceva's we know that
  $$\frac{BX}{XC}\cdot\frac{CY}{YA}\cdot\frac{AZ}{ZB}=1.$$
  By relfection about the midpoint we know that $BX = X'C$ and $BX' = XC.$ Analgous arguments can be applied to the other sides of the triangle. Plugging these into our first equation gives us
  $$\frac{X'C}{BX'}\cdot\frac{Y'A}{CY'}\cdot\frac{Z'B}{AZ'}=1,$$ hence we are done.
\end{proof}

\begin{problem*}
  [4.21]
  Check that if $Q$ is the isogonal conjugate of $P$, then $P$ is the isogonal conjugate of $Q$.
\end{problem*}
\begin{proof}
  This is trivial by the reflexive property of equality. For instance, if $\dang BAP = \dang P^*AC,$ then $\dang P^*AC = \dang BAP.$
\end{proof}

\begin{theorem*}
  [4.22, Isogonal Ratios]
  Let $D$ and $E$ be points on $\ol{BC}$ so that $\ol{AD}$ and $\ol{AE}$ are isogonal. Then
  $$\frac{BD}{DC}\cdot\frac{BE}{EC}=\left(\frac{AB}{AC}\right)^2.$$
\end{theorem*}
\begin{proof}
  By the Ratio Lemma, we have that
  $$\frac{BD}{DC} = \frac{AB}{AC}\cdot\frac{\sin\angle BAD}{\sin\angle DAC}$$ and $$\frac{BE}{EC} = \frac{AB}{AC}\cdot\frac{\sin\angle BAE}{\sin\angle EAC}.$$
  Multiplying the two equations together gives us
  $$\frac{BD}{DC}\cdot\frac{BE}{EC}=\left(\frac{AB}{AC}\right)^2,$$ as desired. (Note that $\angle BAD = \angle EAC$ and $\angle BAE = \angle DAC$ by the definition of being isogonal, so the $\sin$ simplify to 1.)
\end{proof}

\begin{problem*}
  [4.23]
  What is the isogonal conjugate of a triangle’s circumcenter?
\end{problem*}
\begin{proof}
  We claim this point is the orthocenter, $H$. Note that $$\angle BAH = 90\degree - \angle CBA = 90\degree - \frac12\angle COA = \angle OAC.$$
  Analagous arguments can be used for the other vertices of $\triangle ABC,$ hence we are done.
\end{proof}

\begin{problem*}
  [4.25]
  Show that
  $$\frac{CM}{MB} = \frac{\sin\angle B \sin\angle BAX}{\sin\angle C \sin\angle CAX} =1.$$
\end{problem*}
\begin{proof}
  By Law of Sines we have
  $$\frac{MB}{\sin\angle MAB} = \frac{MB}{\sin\angle CAX} = \frac{AM}{\sin\angle B}$$
  and $$\frac{CM}{\sin\angle CAM} = \frac{CM}{\sin\angle BAX} = \frac{AM}{\sin\angle C}.$$
  Solving for $AM$ and combining the two equations gives us
  $$\frac{CM}{MB} = \frac{\sin\angle B \sin\angle BAX}{\sin\angle C \sin\angle CAX}.$$
  Now note that $\angle ABX = 180\degree - \angle C$ and $\angle ACX = 180\degree - \angle B$ by the Tangent Criterion. Also note that $\sin\theta = \sin(180\degree - \theta).$ Thus by the Law of Sines we have
  $$\frac{AX}{\sin\angle ABX} = \frac{AX}{\sin\angle C} = \frac{BX}{\sin\angle BAX}$$
  and $$\frac{AX}{\sin\angle ACX} = \frac{AX}{\sin\angle B} = \frac{CX}{\sin\angle CAX}.$$
  Now noting that $BX = CX$ and combining the equations we get
  $$\frac{\sin\angle B \sin\angle BAX}{\sin\angle C \sin\angle CAX} = 1,$$ as desired.
\end{proof}
\begin{problem*}
  [4.28]
  Verify (d) of Lemma 4.26.
\end{problem*}
\begin{proof}
  Recall that what we want to show is
  $$\frac{AB}{BK}=\frac{AC}{CK}.$$
  Note that $MC=MB$ since $M$ is the midpoint of $BC.$ 
  By using property (b) of Lemma 4.26 twice we have
  $$\frac{AB}{BK}=\frac{AM}{MC} = \frac{AM}{MB}=\frac{AC}{CK}.$$ Hence, we are done.
\end{proof}
\begin{problem*}
  [4.29]
  Show that (f) of Lemma 4.26 follows (with some effort) from (d).
\end{problem*}
\begin{proof}
  (NOT ENTIRELY ORIGINAL) We will first show that $\ol{BC}$ is the $B$-symmedian of $\triangle BAK,$ and the proof regarding the $C$-symmedian will follow analagously. Let $N$ be the point that bisects $AK.$ It suffices to show that $\angle KBN = \angle B,$ since this would satisfy our isogonality requirement for the symmedian. Note that $\angle NKB = \angle AKB = \angle ACB = \angle C.$ Thus, it now suffices to show that $\angle BNK = \angle A.$ Note that $\angle BNK = \angle BNX = \angle BCX = \angle A,$ where the second equality holds by (e) making $B,C,N,X$ concylic and the third equality holds by $\ol{CX}$ being tangent to $(ABC).$ Hence, we're done.
\end{proof}
\begin{problem*}
  [4.31]
  Show that this homothety takes $K$ to $M$, and in particular that $T,K,$ and $M$ are collinear.
\end{problem*}
\begin{proof}
  Note that by tangency, $PK \perp AB.$ By midpoints, $OM \perp AB.$ Thus $PK \parallel OM.$ Also note that $TP=PK$ and $TO=OM,$ thus $\triangle TPK \sim \triangle TOM.$ Thus the homothety centered at $T$ sends $P$ to $O$ and $K$ to $M.$ This also proves that $T,K,M$ are collinear. 
\end{proof}
\begin{problem*}
  [4.32]
  Show that $\triangle TMB \sim \triangle BMK.$
\end{problem*}
\begin{proof}
  By angle chasing we have
  $$\angle MTB = \angle MAB = \angle MBA = \angle MBK.$$
  We also have $\angle TMB = \angle BMK,$ thus $\triangle TMB \sim \triangle BMK$ by $AA\sim$.
\end{proof}
\begin{problem*}
  [4.34, Curvilinear Incircles]
  Prove that the points $C,L,I,T$ are concyclic.
\end{problem*}
\begin{proof}
  We want to show that $\dang TCI = \dang TCM = \dang TLK.$ This is equivalent to showing that the arc measures of $\overset{\frown}{TK}$ and $\overset{\frown}{TM}$ are equal. Since there is a homothety centered at $T$ that sends $K$ to $M,$ the circle $\omega$ will be sent to the circle $\Omega,$ so the two arc measures are equal. Hence, we are done.
\end{proof}
\begin{problem*}
  [4.35, Curvilinear Incircles] Show that \(\triangle MKI \sim \triangle MIT\), and that the triangles are oppositely oriented. 
\end{problem*}
\begin{proof}
  Trivially \(\dang KMI = \dang TMI\). We would like to show that \(\dang IKM = \dang MIT\) to finish off with \(AA \sim\). Note that 
  \begin{align*}
    \dang IKM &= \dang IKT \\
    &= \dang LKT \\
    &= \dang CLT \quad \text{(Tangency Criterion)}\\
    &= \dang CIT \quad (C,L,I,T \text{ are concyclic})\\
    &= \dang MIT.
  \end{align*}
  Note that \(\dang MIT = -\dang MKI,\) so the two triangles are oppositely oriented.
\end{proof}
\begin{problem*}
  [4.37, Mixtilinear Incircles]
  Using the fact that \(I\) lies on \(\ol{KL}\), check that \(I\) is in fact the midpoint of \(\ol{KL}\). 
\end{problem*}
\begin{proof}
  Note that \(\angle KAI = \angle LAI\) because $I$ is the incenter, \(\angle AKI = \angle ALI\) by tangency, and \(AK = AL\) again by tangency, so \(\triangle KAI \cong \triangle LAI\) by \(ASA\cong\). This implies that \(KI = LI,\) and we are done.
\end{proof}
\begin{problem*}
  [4.38, Mixtilinear Incircles]
  Prove that \(\angle ATK = \angle LTI\).
\end{problem*}
\begin{proof}
  By angle chase we have
  \[\angle LTI = \angle LCI = \angle ACM_C = \angle ATM_C = \angle ATK.\]
\end{proof}
\begin{problem*}
  [4.39, Mixtilinear Incircles]
  Prove that $S$ is the midpoint of the arc \(\overset{\frown}{BC}\) containing $A$.
\end{problem*}
\begin{proof}
  By angle chase we have
  \begin{align*}
  \dang SBC &= \dang STC = \dang ITC = \dang ILC = \dang ILA \\
  = \dang AKI &= \dang BKI = \dang BTI = \dang BTS = \dang BCS.
  \end{align*}
  Since we have that \(\angle SBC = \angle SCB,\) this implies the result.
\end{proof}
\begin{problem*}
  [4.41, Hong Kong 1998]
  Let $PQRS$ be a cyclic quadrilateral with $ \angle PSR = 90\degree$ and let $H$ and $K$ be the feet of the altitudes from $Q$ to lines $PR$ and $PS$. Prove that $HK$ bisects $QS$.
\end{problem*}
\begin{proof}
  Note that \(\ol{HK}\) is the Simson Line of \(Q\) with respect to \(\triangle SPR\). Denote by \(L\) the foot of the altitude from \(Q\) to \(\ol{RS}\). Since \(\angle KSL = \angle PSR = 90\degree,\) \(\angle QKS = 90\degree\), and \(\angle SLQ = 90\degree,\) quadrilateral \(KSLQ\) is a rectangle, so its diagonals bisect each other. Thus, \(\ol{KL} = \ol{HK}\) (our Simson Line!) bisects \(SQ\) so we are done.
\end{proof}
\begin{problem*}
  [4.42, USAMO 1998/4]
  Let $ABC$ be a triangle with incenter $I$ and circumcenter $O$. Show that the circumcircles of $\triangle IBC$, $\triangle ICA$ and $\triangle IAB$ lie on a circle with center $O$.
\end{problem*}
\begin{proof}
  Call the circumcenter of $(BIC)$ point $O_A$ and define $O_B$ and $O_C$ similarly. By the Incenter-Excenter Lemma, $O_A$ is the midpoint of arc \(\overset{\frown}{BC}\), $O_B$ is the midpoint of arc \(\overset{\frown}{AC}\), and $O_C$ is the midpoint of arc \(\overset{\frown}{AB}\). This implies that \(O_A, O_B, O_C\) are all on the circumcirle of \(\triangle ABC\) so we are done. (trivial)
\end{proof}
\end{document}